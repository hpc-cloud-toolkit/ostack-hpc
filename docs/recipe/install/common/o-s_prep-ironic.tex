This section we will create generic configuration, required for baremetal provisioning. We will use ironic as a provisioner and nova as a scheduler.
Have selinux in permissive mode
Setenforce 0

Create baremetal admin and baremetal observer role, and restart ironic API

\begin{lstlisting}[language=bash,keywords={}]
openstack role list | grep -i baremetal_admin
role_exists=$?
if [ "${role_exists}" -ne "0" ]; then 
    openstack role create baremetal_admin
fi

openstack role list | grep -i baremetal_observer 
role_exists=$?
if [ "${role_exists}" -ne "0" ]; then
    openstack role create baremetal_observer
fi
systemctl restart openstack-ironic-api
\end{lstlisting}

Install tftp and other packages required for pxe boot via ironinc
Ensure the utilities for baremetal are installed

\begin{lstlisting}[language=bash,keywords={}]
yum install -y tftp-server syslinux-tftpboot xinetd
\end{lstlisting}

Make the directory for tftp and give it the ironic owner

\begin{lstlisting}[language=bash,keywords={}]
mkdir -p /tftpboot
chown -R ironic /tftpboot
\end{lstlisting}

Configure tfpt server

\begin{lstlisting}[language=bash,keywords={}]
(*\#*)Configure tftp 
(*\#*)Configure /etc/xinet.d/tftp
echo "service tftp" > /etc/xinetd.d/tftp
echo "{" >> /etc/xinetd.d/tftp
echo "  protocol        = udp" >> /etc/xinetd.d/tftp
echo "  port            = 69" >> /etc/xinetd.d/tftp
echo "  socket_type     = dgram" >> /etc/xinetd.d/tftp
echo "  wait            = yes" >> /etc/xinetd.d/tftp
echo "  user            = root" >> /etc/xinetd.d/tftp
echo "  server          = /usr/sbin/in.tftpd" >> /etc/xinetd.d/tftp
echo "  server_args     = -v -v -v -v -v --map-file /tftpboot/map-file /tftpboot" >> /etc/xinetd.d/tftp
echo "  disable         = no" >> /etc/xinetd.d/tftp
echo "  (*\#*) This is a workaround for Fedora, where TFTP will listen only on" >> /etc/xinetd.d/tftp
echo "  (*\#*) IPv6 endpoint, if IPv4 flag is not used." >> /etc/xinetd.d/tftp
echo "  flags           = IPv4" >> /etc/xinetd.d/tftp
echo "}" >> /etc/xinetd.d/tftp

(*\#*)Restart the xinetd service
systemctl restart xinetd
    
(*\#*)Copy the PXE linux files to the tftpboot directory we created
cp /var/lib/tftpboot/pxelinux.0 /tftpboot
cp /var/lib/tftpboot/chain.c32 /tftpboot
    
(*\#*)Generate a map file for the PXE files
echo 're ^(/tftpboot/) /tftpboot/\2' > /tftpboot/map-file
echo 're ^/tftpboot/ /tftpboot/' >> /tftpboot/map-file
echo 're ^(^/) /tftpboot/\1' >> /tftpboot/map-file
echo 're ^([^/]) /tftpboot/\1' >> /tftpboot/map-file
\end{lstlisting}


Update Ironic configuration with tftp information. First update controller IP address for tftp server in ironic configuration.

\begin{lstlisting}[language=bash,keywords={}]
sed --in-place "s|(*\#*)tftp_server=\$my_ip|tftp_server=${controller_ip}|" /etc/ironic/ironic.conf
\end{lstlisting}

Update other additional tftp settings in ironic configuration file:


\begin{lstlisting}[language=bash,keywords={}]
sed --in-place "s|(*\#*)tftp_root=/tftpboot|tftp_root=/tftpboot|" /etc/ironic/ironic.conf
sed --in-place "s|(*\#*)ip_version=4|ip_version=4|" /etc/ironic/ironic.conf
sed --in-place "s|(*\#*)automated_clean=true|automated_clean=false|" /etc/ironic/ironic.conf
\end{lstlisting}

Now inform Nova to use ironic for bare metal provisioning, by configuring NOVA.conf

\begin{lstlisting}[language=bash,keywords={}]
sed --in-place "s|(*\#*)scheduler_use_baremetal_filters=false|scheduler_use_baremetal_filters=true|" \
 /etc/nova/nova.conf
\end{lstlisting}

In our sample, we will not use controller node for any compute resource so, lets mark reserved host memory as 0.

\begin{lstlisting}[language=bash,keywords={}]
sed --in-place "s|reserved_host_memory_mb=512|reserved_host_memory_mb=0|" /etc/nova/nova.conf
sed --in-place "s|(*\#*)scheduler_host_subset_size=1|scheduler_host_subset_size=9999999|" /etc/nova/nova.conf
\end{lstlisting}

For cloud-init we need to enable meta data server, which is done via neutron configuration.

\begin{lstlisting}[language=bash,keywords={}]
(*\#*) Enable meta data
(*\#*) Edit /etc/neutron/dhcp_agent.ini
sed --in-place "s|enable_isolated_metadata\ =\ False|enable_isolated_metadata\ =\ True|" \
 /etc/neutron/dhcp_agent.ini
sed --in-place "s|(*\#*)force_metadata\ =\ false|force_metadata\ =\ True|" \ /etc/neutron/dhcp_agent.ini
\end{lstlisting}

We will enable internal dns server to assign host name to the instances as requested by user. 

\begin{lstlisting}[language=bash,keywords={}]
(*\#*)(*\#*)(*\#*)(*\#*)(*\#*)
(*\#*) Enable internal dns for hostname resolution, if it already not set
(*\#*) manipulating configuration file via shell, alternate is to use openstack-config (TODO)
(*\#*)(*\#*)(*\#*)(*\#*)
(*\#*) setup dns domain first
if grep -q "^dns_domain.*openstacklocal$" /etc/neutron/neutron.conf; then
   sed -in-place  "s|^dns_domain.*|dns_domain = oslocal|" /etc/neutron/neutron.conf
else
   if ! grep -q "^dns_domain" neutron.conf; then
       sed -in-place  "s|^(*\#*)dns_domain = openstacklocal$|dns_domain = oslocal|" /etc/neutron/neutron.conf
   fi
fi
(*\#*) configure ml2 dns driver for neutron
ml2file=/etc/neutron/plugins/ml2/ml2_conf.ini
if ! grep -q "^extension_drivers" $ml2file; then
    (*\#*) Assuming there is a place holder in comments, replace that string
    sed -in-place  "s|^(*\#*)extension_drivers.*|extension_drivers = port_security,dns|" $ml2file
else
    (*\#*) Entry is present, check if dns is already present, if not then enable
    if ! grep "^extension_drivers" $ml2file|grep -q dns; then
        current_dns=`grep "^extension_drivers" $ml2file`
        new_dns="$current_dns,dns"
        sed -in-place  "s|^extension_drivers.*|$new_dns|" $ml2file
    fi
fi
\end{lstlisting}

We are pretty much done with initial configuration, so let’s restart all the services at controller node.

\begin{lstlisting}[language=bash,keywords={}]
systemctl restart neutron-dhcp-agent
systemctl restart neutron-openvswitch-agent
systemctl restart neutron-metadata-agent
systemctl restart neutron-server
systemctl restart openstack-nova-scheduler
systemctl restart openstack-nova-compute
systemctl restart openstack-ironic-conductor
\end{lstlisting}
