To build compute node images, we need to install client packages of HPC components. This is accomplished by setting image type to compute. Default image type in hpc elements is "compute".

\begin{lstlisting}[language=bash,keywords={}]
[ctrlr](*\#*) export DIB_HPC_IMAGE_TYPE=compute
\end{lstlisting}

Now enable SLURM resource manager for compute node.

\begin{lstlisting}[language=bash,keywords={}]
[ctrlr](*\#*) DIB_HPC_ELEMENTS+=" hpc-slurm"
\end{lstlisting}

Add optional OpenHPC Components

\begin{lstlisting}[language=bash,keywords={}]
[ctrlr](*\#*)  if [[ ${enable_mrsh} -eq 1 ]];then
[ctrlr](*\#*)         DIB_HPC_ELEMENTS+=" hpc-mrsh"
[ctrlr](*\#*)  fi
\end{lstlisting}

Now create a compute node image with element local-config, dhcp-all-interfaces, devuser, selinux-permisive and all hpc specific elements. Element local-config copies your local environment into image, which is the local users, their password and permissions. Element devuser will create new user specified by environment variable "DIB\_DEV\_USER\_USERNAME". 

\begin{lstlisting}[language=bash,keywords={}]
[ctrlr](*\#*) disk-image-create centos7 vm local-config dhcp-all-interfaces devuser \
 selinux-permissive $DIB_HPC_ELEMENTS -o icloud-hpc-cent7-sms
\end{lstlisting}

It will take a while to build an image. Once image is built, copy it to standard OpenHPC path.

\begin{lstlisting}[language=bash,keywords={}]
[ctrlr](*\#*) chpc_image_sms="$( realpath icloud-hpc-cent7.qcow2)"
[ctrlr](*\#*) mkdir -p $CHPC_CLOUD_IMAGE_PATH
[ctrlr](*\#*) mv -f $chpc_image_user$CHPC_CLOUD_IMAGE_PATH
[ctrlr](*\#*) chpc_image_user=$CHPC_CLOUD_IMAGE_PATH/$(basename $chpc_image_sms)
\end{lstlisting}

