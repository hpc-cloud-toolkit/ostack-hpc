Term "HPC as a Service" refers to an on demand instantiation of HPC Service in a Cloud. This guide presents a simple "HPC cluster" instantiation procedures in an existing OpenStack based System. "HPC as a service" relies on two main principals to instantiate HPC service 
	1. Providing pre-build OS images for compute nodes with HPC optimized software and 
	2. Uses of Cloud-Init to configure and tune HCP services. This recipe provides a simple guides to build HPC optimized OS images, prepare cloud-init recipes and finally instantiate fully functional HPC System using HPC optimized image and cloud-init. 
For HPC System recipe instantiate HPC master node (aka sms node) and HPC compute nodes using pre-configured OS images. The terms master and SMS are used interchangeably in this guide.
OS Images are build using components from OpenHPC software stack. OpenHPC represents an aggregation of a number of common ingredients required to deploy and manage an HPC Linux* cluster including resource management, I/O clients, development tools, and a variety of scientific libraries. These packages have been pre-built with HPC integration in mind using a mix of open-source components. The documentation herein is intended to be reasonably generic,
but uses the underlying motivation of a small, 4-node statefull cluster installation to define a step-by-step
process. Several optional customizations are included and the intent is that these collective instructions can
be modified as needed for local site customizations.
Base Linux Edition: this edition of the guide highlights installation without the use of a companion configuration management system and directly uses distro-provided package management tools for component selection. The steps that follow also highlight specific changes to system configuration files that are required as part of the cluster install process.
