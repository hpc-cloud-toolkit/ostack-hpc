	The cloud-init script for the SMS node is different than the compute node. The SMS node, when instantiated within OpenStack, serves as head node for HPC-as-a-Service, and hosts all services as a SMS node in an independent HPC cluster. The resource manager runs on the SMS node. The SMS node also works as a NFS server and exports users’ home directories and common software to the compute nodes. For more detail on SMS node functionality please refer to OpenHPC documentation http://www.openhpc.community/downloads/.

	In this recipe, we will prepare a cloud-init template script for the SMS node, which then is updated with compute node IP, NTP server, and other environmental variables, just before provisioning. 

	Create an empty chpc\_init file and open it for editing. You can also modify the existing template. Start editing by adding environment variable, which will be updated later, just before provisioning.

% begin_ohpc_run
% ohpc_validation_newline
% ohpc_validation_comment #   PFILEP
% ohpc_command #!/bin/bash
% ohpc_validation_comment # FILE: chpc_sms_init


\begin{lstlisting}[language=bash,keywords={}]
[ctrlr](*\#*) #Ensure the executing shell is in the same directory as the script.
[ctrlr](*\#*) SCRIPTDIR="$( cd "$( dirname "$( readlink -f "${BASH_SOURCE[0]}" )" )" && pwd -P && echo x)"
[ctrlr](*\#*) SCRIPTDIR="${SCRIPTDIR%x}"
[ctrlr](*\#*) cd $SCRIPTDIR
[ctrlr](*\#*) # Get the compute node prefix and number of compute nodes
[ctrlr](*\#*) cnodename_prefix=<update_cnodename_prefix>
[ctrlr](*\#*) num_ccomputes=<update_num_ccomputes>
[ctrlr](*\#*) ntp_server=<update_ntp_server>
[ctrlr](*\#*) sms_name=<update_sms_name>
[ctrlr](*\#*) logger "chpcsmsInit: Entered chpcsmsInit"

\end{lstlisting} 
 % end_ohpc_run

	Now, setup "nfs" share for cloud-init and files you want to send to compute nodes.

% begin_ohpc_run
% ohpc_validation_newline

\begin{lstlisting}[language=bash,keywords={}]
[ctrlr](*\#*) # setup cloudinit directory
[ctrlr](*\#*) chpcInitPath=/opt/ohpc/admin/cloud_hpc_init
[ctrlr](*\#*) # create directory if it is not present
[ctrlr](*\#*) mkdir -p $chpcInitPath
[ctrlr](*\#*) chmod 700 $chpcInitPath
[ctrlr](*\#*) # Copy public ssh key to shared drive
[ctrlr](*\#*) _ssh_path=/root/.ssh
[ctrlr](*\#*) if [ ! -e "$_ssh_path/hpcasservice" ]; then
[ctrlr](*\#*) 
[ctrlr](*\#*) 	if [ ! -d "$_ssh_path" ]; then
[ctrlr](*\#*) 		install -d -m 700 $_ssh_path
[ctrlr](*\#*) 	fi
[ctrlr](*\#*) 	ssh-keygen -t dsa -f $_ssh_path/hpcasservice -N '' -C "HPC Cluster key" > /dev/null 2>&1
[ctrlr](*\#*) 	cat $_ssh_path/hpcasservice.pub >> $_ssh_path/authorized_keys
[ctrlr](*\#*) 	chmod 0600 $_ssh_path/authorized_keys
[ctrlr](*\#*) fi
[ctrlr](*\#*) 	#update config
[ctrlr](*\#*) if [ ! -e "$_ssh_path/config" ]; then
[ctrlr](*\#*) 	echo "Host *" > $_ssh_path/config
[ctrlr](*\#*) 	echo "    IdentityFile ~/.ssh/hpcasservice" >> $_ssh_path/config
[ctrlr](*\#*) 	echo "    StrictHostKeyChecking=no" >> $_ssh_path/config
[ctrlr](*\#*) fi
[ctrlr](*\#*) cp -fpr $_ssh_path/authorized_keys $chpcInitPath
[ctrlr](*\#*) #Change file permissions in /etc/ssh to fix ssh into sms nodes
[ctrlr](*\#*) chmod 0600 /etc/ssh/ssh_host_*_key

\end{lstlisting} 
 % end_ohpc_run

	Share "/home, "/opt/ohpc/pub", and "/opt/ohpc/admin/cloud\_hpc\_initi" over "nfs"

% begin_ohpc_run
% ohpc_validation_newline

\begin{lstlisting}[language=bash,keywords={}]
[ctrlr](*\#*) # export CloudInit Path to nfs share
[ctrlr](*\#*) cat /etc/exports | grep "$chpcInitPath"
[ctrlr](*\#*) chpcInitPath_exported=$?
[ctrlr](*\#*) 
[ctrlr](*\#*) if [ "${chpcInitPath_exported}" -ne "0" ]; then
[ctrlr](*\#*)    echo "$chpcInitPath *(rw,no_subtree_check,no_root_squash)" >> /etc/exports
[ctrlr](*\#*) fi
[ctrlr](*\#*) # share /home from SMS node
[ctrlr](*\#*) if ! grep "^/home" /etc/exports; then
[ctrlr](*\#*)     echo "/home *(rw,no_subtree_check,fsid=10,no_root_squash)" >> /etc/exports
[ctrlr](*\#*) fi
[ctrlr](*\#*) # share /opt/ from SMS node
[ctrlr](*\#*) if ! grep "^/opt/ohpc/pub" /etc/exports; then
[ctrlr](*\#*)     echo "/opt/ohpc/pub *(ro,no_subtree_check,fsid=11)" >> /etc/exports
[ctrlr](*\#*) fi
[ctrlr](*\#*) exportfs -a
[ctrlr](*\#*) # Restart nfs
[ctrlr](*\#*) systemctl restart nfs
[ctrlr](*\#*) systemctl enable nfs-server
[ctrlr](*\#*) logger "chpcsmsInit: nfs configuration complete, updating remaining HPC configuration" 
\end{lstlisting} 
 % end_ohpc_run

	Configure ntp server on the SMS node, as per the site setting.

% begin_ohpc_run
% ohpc_validation_newline

\begin{lstlisting}[language=bash,keywords={}]
[ctrlr](*\#*) # configure NTP
[ctrlr](*\#*) systemctl enable ntpd
[ctrlr](*\#*) if [[ ! -z "$ntp_server" ]]; then
[ctrlr](*\#*)    echo "server ${ntp_server}" >> /etc/ntp.conf
[ctrlr](*\#*) fi
[ctrlr](*\#*) systemctl restart ntpd
[ctrlr](*\#*) systemctl enable ntpd.service
[ctrlr](*\#*) # time sync
[ctrlr](*\#*) ntpstat
[ctrlr](*\#*) logger "chpcInit:ntp configuration done"
\end{lstlisting} 
 % end_ohpc_run

