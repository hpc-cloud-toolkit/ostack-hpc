\documentclass[letterpaper]{article}
\usepackage{./common/ohpc-doc}
\setcounter{secnumdepth}{5}
\setcounter{tocdepth}{5}

% Include git variables
%%% This file has been generated by the vc bundle for TeX.
%%% Do not edit this file!
%%%
%%% Define Git specific macros.
\gdef\GITHash{d964afc2677677407c5fe05d77ecc66d079c7b60}%
\gdef\GITAbrHash{d964afc}%
\gdef\GITParentHashes{448a2318025fdcf51e64cda93b587f17c18571be}%
\gdef\GITAbrParentHashes{448a231}%
\gdef\GITAuthorName{Reese Baird}%
\gdef\GITAuthorEmail{reese.baird@intel.com}%
\gdef\GITAuthorDate{2017-04-06 11:17:28 -0700}%
\gdef\GITCommitterName{Karl W. Schulz}%
\gdef\GITCommitterEmail{karl.w.schulz@intel.com}%
\gdef\GITCommitterDate{2017-04-13 16:41:07 -0500}%
%%% Define generic version control macros.
\gdef\VCRevision{\GITAbrHash}%
\gdef\VCAuthor{\GITAuthorName}%
\gdef\VCDateRAW{2017-04-06}%
\gdef\VCDateISO{2017-04-06}%
\gdef\VCDateTEX{2017/04/06}%
\gdef\VCTime{11:17:28 -0700}%
\gdef\VCModifiedText{\textcolor{red}{with local modifications!}}%
%%% Assume clean working copy.
\gdef\VCModified{0}%
\gdef\VCRevisionMod{\VCRevision}%


% Define Base OS and other local macros
\newcommand{\baseOS}{CentOS7.3}
\newcommand{\OSRepo}{CentOS\_7.3}
\newcommand{\OSTree}{CentOS\_7}
\newcommand{\OSTag}{el7}
\newcommand{\baseos}{centos7.3}
\newcommand{\provisioner}{Openstack}
\newcommand{\rms}{SLURM}
\newcommand{\arch}{x86\_64}
\newcommand{\clean}{yum clean expire-cache}
\newcommand{\chrootclean}{yum --installroot=\$CHROOT clean expire-cache}
\newcommand{\install}{yum -y install}
\newcommand{\chrootinstall}{yum -y --installroot=\$CHROOT install}
\newcommand{\groupinstall}{yum -y groupinstall}
\newcommand{\groupchrootinstall}{yum -y --installroot=\$CHROOT groupinstall}
\newcommand{\upgrade}{yum -y upgrade}
\newcommand{\chrootupgrade}{yum -y --installroot=\$CHROOT upgrade}

\newcommand{\ChTwoCMDOne}{CHPC\_CLOUD\_IMAGE\_PATH=/opt/ohpc/admin/images/cloud/ }
\newcommand{\ChTwoOneCMDOne}{yum -y install diskimage-builder PyYAML }
\newcommand{\ChTwoOneCMDTwo}{yum -y install parted }
%\newcommand{Ch2.1-CMD3}{ yum -y install <DIB patch> }
%\newcommand{Ch2.2-CMD1}{ export DIB\_DEV\_USER\_USERNAME=chpc }
%\newcommand{Ch2.2-CMD2}{ export DIB\_DEV\_USER\_PASSWORD=intel8086 }
%\newcommand{Ch2.2-CMD3}{ export DIB\_DEV\_USER\_PWDLESS\_SUDO=1 }
%\newcommand{Ch2.2-CMD4}{ export ELEMENTS_PATH="\$(realpath ../../dib/hpc/elements)" }
%\newcommand{Ch2.2-CMD5}{ export DIB\_HPC\_FILE\_PATH="\$(realpath ../../dib/hpc/hpc-files/)" }
%\newcommand{Ch2.2-CMD6}{ export DIB\_HPC\_BASE=\"ohpc" }
%\newcommand{Ch2.2-CMD7}{ yum -y install \${ohpc_pkg} }
%\newcommand{Ch2.2-CMD8}{ export DIB\_HPC\_OHPC\_PKG=\${ohpc_pkg} }
%\newcommand{Ch2.2-CMD9}{ DIB\_HPC\_ELEMENTS="hpc-env-base" }
%\newcommand{Ch2.3-CMD1}{ DIB\_YUM\_REPO\_CONF }
%\newcommand{Ch2.3-CMD2}{ yum -y install git }
%\newcommand{Ch2.3-CMD3}{ disk-image-create ironic-agent centos7 -o icloud-hpc-deploy-c7 }
%\newcommand{Ch2.3-CMD4}{ chpc\_image\_deploy\_kernel="\$( realpath icloud-hpc-depl\oy-c7.kernel)" }
%\newcommand{Ch2.3-CMD5}{ chpc\_image\_deploy\_ramdisk="\$( realpath icloud-hpc-deploy-c7.initramfs)" }
%\newcommand{Ch2.3-CMD6}{ mkdir -p \$CHPC\_CLOUD\_IMAGE\_PATH/ }
%\newcommand{Ch2.3-CMD7}{ sudo mv -f \$chpc\_image\_deploy\_kernel \$CHP\_CLOUD\_IMAGE\_PATH/ }
%\newcommand{Ch2.3-CMD8}{ chpc\_image\_deploy\_kernel=\$CHPC\_CLOUD\_IMAGE\_PATH/\$(basename \$chpc\_image\_deploy\_kernel) }
%\newcommand{Ch2.3-CMD9}{ sudo mv -f \$chpc\_image\_deploy\_ramdisk \$CHPC\_CLOUD\_IMAGE\_PATH/ }
%\newcommand{Ch2.3-CMD10}{ chpc\_image\_deploy\_ramdisk=\$CHPC\_CLOUD\_IMAGE\_PATH/\$(basename \$chpc\_image\_deploy\_ramdisk) }
%\newcommand{Ch2.4.1-CMD1}{ export DIB\_HPC\_IMAGE\_TYPE=sms }
%\newcommand{Ch2.4.1-CMD2}{ DIB\_HPC\_ELEMENTS+=" hpc-slurm" }
%\newcommand{Ch2.4.1-CMD3}{ if [[ \$\{enable\_mrsh\} -eq 1 ]];then DIB\_HPC\_ELEMENTS+=" hpc-mrsh"; fi }
%\newcommand{Ch2.4.1-CMD4}{ export DIB\_HPC\_COMPILER="gnu" }
%\newcommand{Ch2.4.1-CMD5}{ export DIB\_HPC\_MPI="openmpi mvapich2" }
%\newcommand{Ch2.4.1-CMD6}{ export DIB\_HPC\_PERF\_TOOLS="perftools" }
%\newcommand{Ch2.4.1-CMD7}{ export DIB\_HPC\_3RD\_LIBS="serial-libs parallel-libs io-libs python-libs runtimes" }
%\newcommand{Ch2.4.1-CMD8}{ DIB\_HPC\_ELEMENTS+=" hpc-dev-env" }
%\newcommand{Ch2.4.1-CMD9}{ disk-image-create centos7 vm local-config dhcp-all-interfaces devuser selinux-permissive \$DIB\_HPC\_ELEMENTS -o icloud-hpc-cent7-sms }
%\newcommand{Ch2.4.1-CMD10}{ chpc\_image\_sms="\$( realpath icloud-hpc-cent7.qcow2)" }
%\newcommand{Ch2.4.1-CMD11}{ mkdir -p \$CHPC\_CLOUD\_IMAGE\_PATH }
%\newcommand{Ch2.4.1-CMD12}{ mv -f \$chpc\_image\_sms \$CHPC\_CLOUD\_IMAGE\_PATH }
%\newcommand{Ch2.4.1-CMD13}{ chpc\_image\_sms=\$CHPC\_CLOUD\_IMAGE\_PATH/\$(basename \$chpc\_image\_sms) }
%\newcommand{Ch2.4.2-CMD1}{ export DIB\_HPC\_IMAGE\_TYPE=compute }
%\newcommand{Ch2.4.2-CMD2}{ DIB\_HPC\_ELEMENTS+=" hpc-slurm" }
%\newcommand{Ch2.4.2-CMD3}{ if [[ \${enable\_mrsh} -eq 1 ]];then DIB\_HPC\_ELEMENTS+=" hpc-mrsh"; fi }
%\newcommand{Ch2.4.2-CMD4}{ disk-image-create centos7 vm local-config dhcp-all-interfaces devuser selinux-permissive \$DIB\_HPC\_ELEMENTS -o icloud-hpc-cent7-sms }
%\newcommand{Ch2.4.2-CMD5}{ chpc\_image\_sms="\$( realpath icloud-hpc-cent7.qcow2)" }
%\newcommand{Ch2.4.2-CMD6}{ mkdir -p \$CHPC\_CLOUD\_IMAGE\_PATH }
%\newcommand{Ch2.4.2-CMD7}{ mv -f \$chpc\_image\_user \$CHPC\_CLOUD\_IMAGE\_PATH }
%\newcommand{Ch2.4.2-CMD8}{ chpc\_image\_user=\$CHPC\_CLOUD\_IMAGE\_PATH/\$(basename \$chpc_image_sms) }
%\newcommand{Ch3.1-CMD1}{ chpcInitPath=/opt/ohpc/admin/cloud_hpc_init }
%\newcommand{Ch3.1-CMD2}{ logger "chpcInit: Updating Compute Node with HPC configuration" }
%\newcommand{Ch3.1-CMD3}{ cat /etc/rsyslog.conf | grep "<sms_ip>:514" }
%\newcommand{Ch3.1-CMD4}{ rsyslog_set=\$? }
%\newcommand{Ch3.1-CMD5}{ if [ "\${rsyslog_set}" -ne "0" ]; then echo "*.* @<sms_ip>:514" >> /etc/rsyslog.conf; fi }
%\newcommand{Ch3.1-CMD6}{ systemctl restart rsyslog }
%\newcommand{Ch3.1-CMD7}{ logger "chpcInit: rsyslog configuration complete, updating remaining HPC configuration" }
%\newcommand{Ch3.1-CMD8}{ cat /etc/fstab | grep "<sms_ip>:/home" }
%\newcommand{Ch3.1-CMD9}{ home\_exists=\$? }
%\newcommand{Ch3.1-CMD10}{ if [ "\${home\_exists}" -ne "0" ]; then echo "<sms\_ip>:/home /home nfs nfsvers=3,rsize=1024,wsize=1024,cto 0 0" >> /etc/fstab; fi }
%\newcommand{Ch3.1-CMD11}{ cat /etc/fstab | grep "<sms\_ip>:/opt/ohpc/pub" }
%\newcommand{Ch3.1-CMD12}{ ohpc\_pub\_exists=\$? }
%\newcommand{Ch3.1-CMD13}{ if [ "\${ohpc\_pub\_exists}" -ne "0" ]; then echo "<sms\_ip>:/opt/ohpc/pub /opt/ohpc/pub nfs nfsvers=3 0 0" >> /etc/fstab; if [ -e /opt/ohpc/pub ]; then echo "chpcInit: [WARNING] /opt/ohpc/pub already exists!!"; fi; fi }
%\newcommand{Ch3.1-CMD14}{ mkdir -p /opt/ohpc/pub }
%\newcommand{Ch3.1-CMD15}{ mount /home }
%\newcommand{Ch3.1-CMD16}{ mount /opt/ohpc/pub }
%\newcommand{Ch3.1-CMD17}{ cat /etc/fstab | grep "sms\_ip:\$chpcInitPath" }
%\newcommand{Ch3.1-CMD18}{ CloudHPCInit\_exist=\$? }
%\newcommand{Ch3.1-CMD19}{ if [ "\${CloudHPCInit_exist}" -ne "0" ]; then echo "<sms\_ip>:\$chpcInitPath \$chpcInitPath nfs nfsvers=3 0 0" >> /etc/fstab; fi }
%\newcommand{Ch3.1-CMD20}{ mkdir -p \$chpcInitPath }
%\newcommand{Ch3.1-CMD21}{ mount \$chpcInitPath }
%\newcommand{Ch3.1-CMD22}{ systemctl restart nfs }
%\newcommand{Ch3.1-CMD23}{ have ntp sync with sms node.  }
%\newcommand{Ch3.1-CMD24}{ systemctl enable ntpd }
%\newcommand{Ch3.1-CMD25}{ cat /etc/ntp.conf | grep "server <sms_ip>" }
%\newcommand{Ch3.1-CMD26}{ ntp_server\_exists=\$? }
%\newcommand{Ch3.1-CMD27}{ if [ "\${ntp\_server\_exists}" -ne "0" ]; then echo "server <sms\_ip>" >> /etc/ntp.conf; fi }
%\newcommand{Ch3.1-CMD28}{ systemctl restart ntpd }
%\newcommand{Ch3.1-CMD29}{ if [ -d \$chpcInitPath ]; then cp -f -L \$chpcInitPath/slurm.conf /etc/slurm/slurm.conf; cp -f -L \$chpcInitPath/passwd /etc/passwd; cp -f -L \$chpcInitPath/group /etc/group; cp -f -L \$chpcInitPath/shadow /etc/shadow; cp -f -L \$chpcInitPath/slurm.conf /etc/slurm/slurm.conf; cp -f -L \$chpcInitPath/slurm /etc/pam.d/slurm; cp -f -L \$chpcInitPath/munge.key /etc/munge/munge.key; cp -f -L \$chpcInitPath/hosts /etc/hosts; systemctl enable munge; systemctl restart munge; systemctl enable slurmd; systemctl restart slurmd; else logger "chpcInit:ERROR: cannot stat nfs shared /opt directory, cannot copy HPC system files"; fi }
%\newcommand{Ch3.1-CMD30}{ cc\_ipaddrs=(`hostname -I`) }
%\newcommand{Ch3.1-CMD31}{ for cc\_ipaddr in \${cc\_ipaddrs[@]}; do cat /etc/hosts | grep \${cc\_ipaddr} > /dev/null; result=\$?; if [ "\$result" -eq "0" ]; then cc\_hostname=`cat /etc/hosts | grep \${cc\_ipaddr} | cut -d\$'\t' -f2`; break; fi; done }
%\newcommand{Ch3.1-CMD32}{ if [ -z "\${cc\_hostname}" ]; then logger "chpcInit:ERROR: No resolved hostname found for any IP address in /etc/hosts"; exit 1; fi }
%\newcommand{Ch3.1-CMD33}{ if [ \$(hostname) != \${cc\_hostname} ]; then hostnamectl set-hostname \${cc\_hostname}; fi }
%\newcommand{Ch3.1-CMD34}{ systemctl enable munge }
%\newcommand{Ch3.1-CMD35}{ systemctl restart munge }
%\newcommand{Ch3.1-CMD36}{ systemctl enable slurmd }
%\newcommand{Ch3.1-CMD37}{ systemctl restart slurmd }
%\newcommand{Ch3.1-CMD38}{ chmod 0600 /etc/ssh/ssh\_host\_*\_key }
%\newcommand{Ch3.2-CMD1}{ cnodename\_prefix=<update\_cnodename\_prefix> }
%\newcommand{Ch3.2-CMD2}{ num\_ccomputes=<update\_num\_ccomputes> }
%\newcommand{Ch3.2-CMD3}{ ntp\_server=<update\_ntp\_server> }
%\newcommand{Ch3.2-CMD4}{ sms\_name=<update\_sms\_name> }
%\newcommand{Ch3.2-CMD5}{ chpcInitPath=/opt/ohpc/admin/cloud\_hpc\_init }
%\newcommand{Ch3.2-CMD6}{ mkdir -p \$chpcInitPath }
%\newcommand{Ch3.2-CMD7}{ chmod 700 \$chpcInitPath }
%\newcommand{Ch3.2-CMD8}{ sudo cp -fpr /etc/passwd \$chpcInitPath }
%\newcommand{Ch3.2-CMD9}{ sudo cp -fpr /etc/shadow \$chpcInitPath }
%\newcommand{Ch3.2-CMD10}{ sudo cp -fpr /etc/group \$chpcInitPath }
%\newcommand{Ch3.2-CMD11}{ cat /etc/exports | grep "\$chpcInitPath" }
%\newcommand{Ch3.2-CMD12}{ chpcInitPath_exported=\$? }
%\newcommand{Ch3.2-CMD13}{ if [ "\${chpcInitPath\_exported}" -ne "0" ]; then echo "\$chpcInitPath *(rw,no\_subtree\_check,no\_root\_squash)" >> /etc/exports; fi }
%\newcommand{Ch3.2-CMD14}{ if ! grep "^/home" /etc/exports; then echo "/home *(rw,no\_subtree\_check,fsid=10,no\_root\_squash)" >> /etc/exports; fi }
%\newcommand{Ch3.2-CMD15}{ if ! grep "^/opt/ohpc/pub" /etc/exports; then echo "/opt/ohpc/pub *(ro,no\_subtree\_check,fsid=11)" >> /etc/exports; fi }
%\newcommand{Ch3.2-CMD16}{ exportfs -a }
%\newcommand{Ch3.2-CMD17}{ systemctl restart nfs }
%\newcommand{Ch3.2-CMD18}{ systemctl enable nfs-server }
%\newcommand{Ch3.2-CMD19}{ logger "chpcInit: nfs configuration complete, updating remaining HPC configuration" }
%\newcommand{Ch3.2-CMD20}{ systemctl enable ntpd }
%\newcommand{Ch3.2-CMD21}{ if [[ ! -z "\$ntp\_server" ]]; then echo "server \${ntp\_server}" >> /etc/ntp.conf; fi }
%\newcommand{Ch3.2-CMD22}{ systemctl restart ntpd }
%\newcommand{Ch3.2-CMD23}{ systemctl enable ntpd.service }
%\newcommand{Ch3.2-CMD24}{ ntpstat }
%\newcommand{Ch3.2-CMD25}{ logger "chpcInit:ntp configuration done" }
%\newcommand{Ch3.2-CMD26}{ perl -pi -e "s/ControlMachine=\\S+/ControlMachine=\${sms\_name}/" /etc/slurm/slurm.conf }
%\newcommand{Ch3.2-CMD27}{ perl -pi -e "s/^NodeName=(\\S+)/NodeName=\${cnodename\_prefix}[1-\${num\_ccomputes}]/" /etc/slurm/slurm.conf }
%\newcommand{Ch3.2-CMD28}{ perl -pi -e "s/^PartitionName=normal Nodes=(\\S+)/PartitionName=normal Nodes=\${cnodename\_prefix}[1-\${num\_ccomputes}]/" /etc/slurm/slurm.conf }
%\newcommand{Ch3.2-CMD29}{ cp -fpr -L /etc/slurm/slurm.conf \$chpcInitPath }
%\newcommand{Ch3.2-CMD30}{ cp -fpr -L /etc/pam.d/slurm \$chpcInitPath }
%\newcommand{Ch3.2-CMD31}{ cp -fpr -L /etc/munge/munge.key \$chpcInitPath }
%\newcommand{Ch3.2-CMD32}{ systemctl enable munge }
%\newcommand{Ch3.2-CMD33}{ systemctl restart munge }
%\newcommand{Ch3.2-CMD34}{ systemctl enable slurmctld }
%\newcommand{Ch3.2-CMD35}{ systemctl restart slurmctld }
%\newcommand{Ch3.2-CMD36}{ logger "chpcInit:slurm configuration done" }
%\newcommand{Ch3.2-CMD37}{ chmod 0600 /etc/ssh/ssh\_host\_*\_key }
%\newcommand{Ch3.2-CMD38}{ cat /etc/services | grep mshell }
%\newcommand{Ch3.2-CMD39}{ mshell_exists=\$? }
%\newcommand{Ch3.2-CMD40}{ if [ "\${mshell\_exists}" -ne "0" ]; then echo "mshell          21212/tcp   (*\#*) mrshd" >> /etc/services; fi }
%\newcommand{Ch3.2-CMD41}{ cat /etc/services | grep mlogin }
%\newcommand{Ch3.2-CMD42}{ mlogin\_exists=\$? }
%\newcommand{Ch3.2-CMD43}{ if [ "\${mlogin\_exists}" -ne "0" ]; then  echo "mlogin         541/tcp    (*\#*) mrlogind" >> /etc/services; fi }
%\newcommand{Ch3.2-CMD44}{ sed -i -- 's/all: @adm,@compute/compute: cc[1-\${num\_ccomputes}]\\n&/' /etc/clustershell/groups.d/local.cfg }
%\newcommand{Ch3.3.1-CMD1}{ cat /etc/services | grep mshell }
%\newcommand{Ch3.3.1-CMD2}{ mshell_exists=\$? }
%\newcommand{Ch3.3.1-CMD3}{ if [ "\${mshell\_exists}" -ne "0" ]; then echo "mshell          21212/tcp                  (*\#*) mrshd" >> /etc/services; fi }
%\newcommand{Ch3.3.1-CMD4}{ cat /etc/services | grep mlogin }
%\newcommand{Ch3.3.1-CMD5}{ mlogin\_exists=\$? }
%\newcommand{Ch3.3.1-CMD6}{ if [ "\${mlogin\_exists}" -ne "0" ]; then echo "mlogin            541/tcp                  (*\#*) mrlogind" >> /etc/services; fi }
%\newcommand{Ch3.3.2-CMD1}{ sed -i -- 's/all: @adm,@compute/compute: cc[1-\${num\_ccomputes}]\\n&/' /etc/cluste\rshell/groups.d/local.cfg }
%\newcommand{Ch3.4-CMD1}{chpcInitPath=/opt/ohpc/admin/cloud\_hpc\_init}
%\newcommand{Ch3.4-CMD2}{ mkdir -p \$chpcInitPath}
%\newcommand{Ch3.4-CMD3}{ sudo cp -fr -L < \${SCRIPTDIR} >/ cloud\_hpc\_init/\${chpc\_base}/* \$chpcInitPath/}
%\newcommand{Ch3.4-CMD4}{ export chpcInit=\$chpcInitPath/chpc\_init}
%\newcommand{Ch3.4-CMD5}{ export chpcSMSInit=\$chpcInitPath/chpc\_sms\_init}
%\newcommand{Ch3.4-CMD6}{ sudo sed -i -e "s/<sms_ip>/\${sms\_ip}/g" \$chpcInit}
%\newcommand{Ch3.4-CMD7}{ sudo sed -i -e "s/<update\_cnodename\_prefix>/\${cnodename\_prefix}/g" \$chpcSMSInit}
%\newcommand{Ch3.4-CMD8}{ sudo sed -i -e "s/<update\_num\_ccomputes>/\${num\_ccomputes}/g" \$chpcSMSInit}
%\newcommand{Ch3.4-CMD9}{ sudo sed -i -e "s/<update\_ntp\_server>/\${controller\_ip}/g" \$chpcSMSInit}
%\newcommand{Ch3.4-CMD10}{ sudo sed -i -e "s/<update\_sms\_name>/\${sms\_name}/g" \$chpcSMSInit}
%\newcommand{Ch3.4-CMD11}{ if [[ \${enable\_mrsh} -eq 1 ]];then cat \$CHPC\_SCRIPTDIR/sms/update\_mrsh >> \$chpcSMSInit; fi}
%\newcommand{Ch3.4-CMD12}{ if [[ \${enable\_clustershell} -eq 1 ]];then cat \$CHPC\_SCRIPTDIR/sms/update\_clustershell >> \$chpcSMSInit; fi}
%\newcommand{Ch4.0-CMD1}{ openstack service list}
%\newcommand{Ch4.0-CMD2}{ openstack project list}
%\newcommand{Ch4.0-CMD2}{unset OS\_SERVICE_TOKEN}
%\newcommand{Ch4.0-CMD2}{export OS\_USERNAME=admin}
%\newcommand{Ch4.0-CMD2}{export OS\_PASSWORD=<>}
%\newcommand{Ch4.0-CMD2}{export OS\_AUTH\_URL=<>}
%\newcommand{Ch4.0-CMD2}{export PS1='[\\u@\\h \\W(keystoane_admin\)]\$ '}
%\newcommand{Ch4.0-CMD2}{export OS\_TENANT\_NAME=admin}
%\newcommand{Ch4.0-CMD2}{export OS\_REGION\_NAME=<> }
%\newcommand{Ch4.1-CMD1}{openstack role list | grep -i baremetal\_admin}
%\newcommand{Ch4.1-CMD2}{role\_exists=\$?}
%\newcommand{Ch4.1-CMD3}{if [ "\${role\_exists}" -ne "0" ]; then openstack role create baremetal\_admin; fi }
%\newcommand{Ch4.1-CMD4}{openstack role list | grep -i baremetal\_observer}
%\newcommand{Ch4.1-CMD5}{role\_exists=\$?}
%\newcommand{Ch4.1-CMD6}{if [ "\${role\_exists}" -ne "0" ]; then openstack role create baremetal\_observer; fi}
%\newcommand{Ch4.1-CMD7}{systemctl restart openstack-ironic-api}
%\newcommand{Ch4.1-CMD8}{yum install -y tftp-server syslinux-tftpboot xinetd}
%\newcommand{Ch4.1-CMD9}{mkdir -p /tftpboot}
%\newcommand{Ch4.1-CMD10}{chown -R ironic /tftpboot}
%\newcommand{Ch4.1-CMD11}{echo "service tftp" > /etc/xinetd.d/tftp; echo "{" >> /etc/xinetd.d/tftp; echo "  protocol        = udp" >> /etc/xinetd.d/tftp; echo "  port            = 69" >> /etc/xinetd.d/tftp; echo "  socket_type     = dgram" >> /etc/xinetd.d/tftp; echo "  wait            = yes" >> /etc/xinetd.d/tftp; echo "  user            = root" >> /etc/xinetd.d/tftp; echo "  server          = /usr/sbin/in.tftpd" >> /etc/xinetd.d/tftp; echo "  server_args     = -v -v -v -v -v --map-file /tftpboot/map-file /tftpboot" >> /etc/xinetd.d/tftp; echo "  disable         = no" >> /etc/xinetd.d/tftp; echo "  (*\#*) This is a workaround for Fedora, where TFTP will listen only on" >> /etc/xinetd.d/tftp; echo "  (*\#*) IPv6 endpoint, if IPv4 flag is not used." >> /etc/xinetd.d/tftp; echo "  flags           = IPv4" >> /etc/xinetd.d/tftp; echo "}" >> /etc/xinetd.d/tftp}
%\newcommand{Ch4.1-CMD12}{systemctl restart xinetd}
%\newcommand{Ch4.1-CMD13}{cp /var/lib/tftpboot/pxelinux.0 /tftpboot}
%\newcommand{Ch4.1-CMD14}{cp /var/lib/tftpboot/chain.c32 /tftpboot}
%\newcommand{Ch4.1-CMD15}{echo 're ^(/tftpboot/) /tftpboot/\2' > /tftpboot/map-file; echo 're ^/tftpboot/ /tftpboot/' >> /tftpboot/map-file; echo 're ^(^/) /tftpboot/\1' >> /tftpboot/map-file; echo 're ^([^/]) /tftpboot/\1' >> /tftpboot/map-file}
%\newcommand{Ch4.1-CMD16}{sed --in-place "s|(*\#*)tftp\_server=\\\$my\_ip|tftp\_server=\${controller\_ip}|" /etc/ironic/ironic.conf}
%\newcommand{Ch4.1-CMD17}{sed --in-place "s|(*\#*)tftp\_root=/tftpboot|tftp\_root=/tftpboot|" /etc/ironic/ironic.conf}
%\newcommand{Ch4.1-CMD18}{sed --in-place "s|(*\#*)ip\_version=4|ip\_version=4|" /etc/ironic/ironic.conf}
%\newcommand{Ch4.1-CMD19}{sed --in-place "s|(*\#*)automated\_clean=true|automated\_clean=false|" /etc/ironic/ironic.conf}
%\newcommand{Ch4.1-CMD20}{sed --in-place "s|(*\#*)scheduler\_use\_baremetal\_filters=false|scheduler\_use\_baremetal\_filters=true|" /etc/nova/nova.conf}
%\newcommand{Ch4.1-CMD21}{sed --in-place "s|reserved\_host\_memory\_mb=512|reserved\_host\_memory\_mb=0|" /etc/nova/nova.conf}
%\newcommand{Ch4.1-CMD22}{sed --in-place "s|(*\#*)scheduler\_host\_subset\_size=1|scheduler\_host\_subset\_size=9999999|" /etc/nova/nova.conf}
%\newcommand{Ch4.1-CMD23}{sed --in-place "s|enable\_isolated\_metadata\ =\ False|enable\_isolated\_metadata\ =\ True|" /etc/neutron/dhcp\_agent.ini}
%\newcommand{Ch4.1-CMD24}{sed --in-place "s|(*\#*)force\_metadata\ =\ false|force\_metadata\ =\ True|" \ /etc/neutron/dhcp_agent.ini}
%\newcommand{Ch4.1-CMD25}{if grep -q "^dns\_domain.*openstacklocal\$" /etc/neutron/neutron.conf; then sed -in-place  "s|^dns\_domain.*|dns\_domain = oslocal|" /etc/neutron/neutron.conf; else ; if ! grep -q "^dns\_domain" neutron.conf; then sed -in-place  "s|^(*\#*)dns\_domain = openstacklocal\$|dns\_domain = oslocal|" /etc/neutron/neutron.conf; fi; fi}
%\newcommand{Ch4.1-CMD26}{ml2file=/etc/neutron/plugins/ml2/ml2\_conf.ini}
%\newcommand{Ch4.1-CMD27}{if ! grep -q "^extension\_drivers" \$ml2file; then sed -in-place  "s|^(*\#*)extension\_drivers.*|extension\_drivers = port\_security,dns|" \$ml2file; else; if ! grep "^extension\_drivers" \$ml2file|grep -q dns; then current\_dns=`grep "^extension\_drivers" \$ml2file`; new\_dns="\$current_dns,dns"; sed -in-place  "s|^extension_drivers.*|\$new\_dns|" \$ml2file; fi; fi}
%\newcommand{Ch4.1-CMD28}{for i in neutron-dhcp-agent neutron-openvswitch-agent neutron-metadata-agent neutron-server openstack-nova-scheduler openstack-nova-compute openstack-ironic-conductor; do systemctl restart \$i; done}
%\newcommand{Ch4.2.1-CMD1}{  SERVICES\_TENANT\_ID=`keystone tenant-list | grep "|\\s*services\\s*|" | awk '{print \$2}'`}
%\newcommand{Ch4.2.1-CMD2}{neutron net-list | grep "|\\s*sharednet1\\s*|"}
%\newcommand{Ch4.2.1-CMD3}{net\_exists=\$?}
%\newcommand{Ch4.2.1-CMD4}{if [ "\${net\_exists}" -ne "0" ]; then neutron net-create --tenant-id \${SERVICES\_TENANT\_ID} sharednet1 --shared --provider:network\_type flat --provider:physical\_network physnet1; fi}
%\newcommand{Ch4.2.1-CMD5}{NEUTRON\_NETWORK\_UUID=`neutron net-list | grep "|\\s*sharednet1\\s*|" | awk '{print \$2}'`}
%\newcommand{Ch4.2.1-CMD6}{neutron subnet-list | grep "|\\s*subnet01\\s*|"}
%\newcommand{Ch4.2.1-CMD7}{subnet\_exists=\$?}
%\newcommand{Ch4.2.1-CMD8}{if [ "\${subnet\_exists}" -ne "0" ]; then neutron subnet-create sharednet1 --name subnet01 --ip-version=4 --gateway=\${controller\_ip} --allocation-pool start=\${cc_subnet\_dhcp\_start},end=\${cc\_subnet\_dhcp\_end} --enable-dhcp \${cc\_subnet\_cidr}; fi}
%\newcommand{Ch4.2.1-CMD9}{NEUTRON\_SUBNET\_UUID=`neutron subnet-list | grep "|\\s*subnet01\\s*|" | awk '{print \$2}'`}
%\newcommand{Ch4.2.1-CMD10}{glance image-list | grep "|\\s*deploy-vmlinuz\\s*|"}
%\newcommand{Ch4.2.1-CMD11}{img\_exists=\$?}
%\newcommand{Ch4.2.1-CMD12}{if [ "\${img\_exists}" -ne "0" ]; then glance image-create --name deploy-vmlinuz --visibility public --disk-format aki --container-format aki < \${chpc\_image\_deploy\_kernel}; fi}
%\newcommand{Ch4.2.1-CMD13}{DEPLOY\_VMLINUZ\_UUID=`glance image-list | grep "|\\s*deploy-vmlinuz\\s*|" | awk '{print \$2}'`}
%\newcommand{Ch4.2.1-CMD14}{glance image-list | grep "|\\s*deploy-initrd\\s*|"}
%\newcommand{Ch4.2.1-CMD15}{img\_exists=\$?}
%\newcommand{Ch4.2.1-CMD16}{if [ "\${img\_exists}" -ne "0" ]; then glance image-create --name deploy-initrd --visibility public --disk-format ari --container-format ari < \${chpc\_image\\_deploy\_ramdisk}; fi}
%\newcommand{Ch4.2.1-CMD17}{DEPLOY\_INITRD\_UUID=\`glance image-list | grep "|\\s*deploy-initrd\\s*|" | awk '{print \$2}}
%\newcommand{Ch4.2.1-CMD18}{nova flavor-list | grep "|\\s*baremetal-flavor\\s*|"}
%\newcommand{Ch4.2.1-CMD19}{flavor\_exists=\$?}
%\newcommand{Ch4.2.1-CMD20}{if [ "\$flavor\_exists" -ne "0" ]; then nova flavor-create baremetal-flavor baremetal-flavor \${RAM_MB} \${DISK_GB} \${CPU}; nova flavor-key baremetal-flavor set cpu_arch=\$ARCH; fi}
%\newcommand{Ch4.2.1-CMD21}{FLAVOR_UUID=`nova flavor-list | grep "|\\s*baremetal-flavor\\s*|" | awk '{print \$2}'`}
%\newcommand{Ch4.2.1-CMD22}{openstack quota set --ram 512000 --cores 1000 --instances 100 admin}
%\newcommand{Ch4.2.1-CMD23}{nova keypair-list | grep "|\\s*ostack_key\\s*|"}
%\newcommand{Ch4.2.1-CMD24}{keypair\_exists=\$?}
%\newcommand{Ch4.2.1-CMD25}{if [ "\${keypair\_exists}" -ne "0" ]; then nova keypair-add --pub-key \${HOME}/.ssh/id_rsa.pub ostack_key; fi}
%\newcommand{Ch4.2.1-CMD26}{KEYPAIR\_NAME=ostack\_key}
%\newcommand{Ch4.2.2-CMD1}{glance image-list | grep "|\\s*sms-image\\s*|"}
%\newcommand{Ch4.2.2-CMD2}{img\_exists=\$?}
%\newcommand{Ch4.2.2-CMD3}{if [ "\${img\_exists}" -ne "0" ]; then glance image-create --name sms-image --visibility public --disk-format qcow2 --container-format bare < \${chpc\_image\_sms}; fi}
%\newcommand{Ch4.2.2-CMD4}{SMS\_DISK\_IMAGE\_UUID=`glance image-list | grep "|\\s*sms-image\\s*|" | awk '{print \$2}'`}
%\newcommand{Ch4.2.2-CMD5}{ironic node-list | grep "|\\s*\${sms_name}\$\\s*|"}
%\newcommand{Ch4.2.2-CMD6}{node\_exists=\$?}
%\newcommand{Ch4.2.2-CMD7}{if [ "\${node\_exists}" -ne "0" ]; then ironic node-create -d pxe\_ipmitool -i deploy\_kernel=\${DEPLOY\_VMLINUZ\_UUID} -i deploy\_ramdisk=\${DEPLOY\_INITRD\_UUID} -i ipmi\_terminal\_port=8023 -i ipmi\_address=\${sms\_bmc} -i ipmi\_username=\${sms\_bmc\_username} -i ipmi\_password=\${sms\_bmc\_password} -p cpus=\${CPU} -p memory\_mb=\${RAM_MB} -p local\_gb=\${DISK\_GB} -p cpu\_arch=\${ARCH} -p capabilities="boot_mode:bios" -n \${sms\_name}; fi}
%\newcommand{Ch4.2.2-CMD8}{SMS\_UUID=`ironic node-list | grep "|\\s*\${sms_name}\\s*|" | awk '{print \$2}'`}
%\newcommand{Ch4.2.2-CMD9}{ironic port-create -n \${SMS\_UUID} -a \${sms\_mac}}
%\newcommand{Ch4.2.2-CMD10}{ironic node-update \$SMS\_UUID add instance\_info/image\_source=\${SMS\_DISK\_IMAGE\_UUID} instance\_info/root\_gb=50}
%\newcommand{Ch4.2.2-CMD11}{neutron port-create sharednet1 --dns\_name \$sms\_name --fixed-ip ip\_address=\$sms\_ip --name \$sms\_name --mac-address \$sms\_mac}
%\newcommand{Ch4.2.2-CMD12}{SMS\_PORT\_ID=`neutron port-list | grep "|\\s*\$sms\_name\\s*|" | awk '{print \$2}'`}
%\newcommand{Ch4.2.3-CMD1}{glance image-list | grep "|\\s*user-image\\s*|"}
%\newcommand{Ch4.2.3-CMD2}{img\_exists=\$?}
%\newcommand{Ch4.2.3-CMD3}{if [ "\${img\_exists}" -ne "0" ]; then glance image-create --name user-image --visibility public --disk-format qcow2 --container-format bare < \${chpc\_image\_user}; fi}
%\newcommand{Ch4.2.3-CMD4}{USER\_DISK\_IMAGE\_UUID=`glance image-list | grep "|\\s*user-image\\s*|" | awk '{print \$2}'`}
%\newcommand{Ch4.2.3-CMD5}{sleep 121}
%\newcommand{Ch4.2.4-CMD1}{echo "nova boot --config-drive true --flavor \${FLAVOR\_UUID} --image \${SMS\_DISK\_IMAGE\_UUID} --key-name \${KEYPAIR\_NAME} --meta role=webservers --user-data=\$chpcSMSInit --nic port-id=\${SMS\_PORT\_ID} \${sms\_name}" > boot\_sms}
%newcommand{Ch4.2.4-CMD2}{nova boot --config-drive true --flavor \${FLAVOR\_UUID} --image \${SMS\_DISK\_IMAGE\_UUID} --key-name \${KEYPAIR\_NAME} --meta role=webservers --user-data=\$chpcSMSInit --nic port-id=\${SMS\_PORT\_ID} \${sms\_name}}
%\newcommand{Ch4.2.4-CMD3}{sleep 15}
%\newcommand{Ch4.2.5-CMD1}{for ((i=0; i < \${num_ccomputes}; i++)); do}
%filename="cn\$((i+1))"; echo "nova boot --config-drive true --flavor \${FLAVOR\_UUID} --image \${USER\_DISK\_IMAGE\_UUID} --key-name \${KEYPAIR\_NAME} --meta role=webservers --user-data=\$chpcInit --nic port-id=\${NEUTRON\_PORT\_ID\_CC[\$i]} \${cnodename\_prefix}\$((i+1))" > boot\_\$filename; nova boot --config-drive true --flavor \${FLAVOR\_UUID} --image \${USER\_DISK\_IMAGE\_UUID} --key-name \${KEYPAIR\_NAME} --meta role=webservers --user-data=\$chpcInit --nic port-id=\${NEUTRON\_PORT\_ID\_CC[\$i]} \${cnodename\_prefix}\$((i+1)); sleep 5; done}


% boolean for os-specific formatting
\toggletrue{isCentOS}
\toggletrue{isCentOS_ww_slurm_x86}
\toggletrue{isx86}

\begin{document}
\graphicspath{{common/figures/}}
\thispagestyle{empty}

% Title Page
% Title page and running header definition

\lhead{ \small {\color{logodarkgrey}\fontfamily{phv}\selectfont { HPC as a Service
    (v\CHPCVersion{})}:  {\baseOS{}/\arch{} + \provisioner{} + \rms{}} } \vspace*{0pt} }

%{\hspace*{4in}\includegraphics[width=1.7in]{ohpc_logo_blue.pdf}}

\vspace*{2cm}
\noindent {\LARGE \color{logodarkgrey} \fontfamily{phv}\selectfont CloudHPC (v\CHPCVersion{})} \vspace*{0.1cm} \\
\noindent {\LARGE \color{logodarkgrey} \fontfamily{phv}\selectfont OHPC on Openstack} \\ 

{\color{logoblue}\noindent\rule{6.15in}{1.2pt}} \\ 

\noindent {\Large \color{logodarkgrey} \fontfamily{phv}\selectfont \baseOS{} Base OS} \\ 

\noindent{\Large\color{logodarkgrey}\fontfamily{phv}\selectfont{\provisioner{}/\rms{}
Edition for Linux*} (\arch{})} \\

{\color{logoblue}\noindent\rule{6.15in}{1.2pt}} \\ \vspace{0.2cm}  

\vspace*{2in}

\noindent{\small \color{black} Document Last Update: 28JUN2017 } \vspace*{0.1cm} \\ 
{\small \color{black} Document Revision: 1.4} \\ \vspace*{0.1cm}

% Disclaimer 
\input{common/legal} 

\newpage
\tableofcontents
\newpage

% Introduction  --------------------------------------------------

\section{Introduction} \label{sec:introduction}
\input{common/install_header}
The term "HPC as a Service" refers to an on demand instantiation of HPC Service in a Cloud environment. This guide presents a simple "HPC cluster" instantiation procedure on an existing OpenStack (Mitaka) system. "HPC as a service" relies on two main principals to instantiate HPC service :


\begin{list}{}
	\item 	1. Providing pre-built OS images for compute nodes with HPC optimized software and 
	\item   2. Use of cloud-init to configure and tune HPC services.
	 
\end{list}
	
This document provides a simple guide to build HPC optimized OS images, prepare cloud-init recipes and finally instantiate a fully functional HPC System using those images and cloud-init. 

Recipes will instantiate and HPC master node (a.k.a. sms node) and HPC compute nodes using pre-configured OpenStack images. The terms "master" and "sms" are used interchangeably in this guide.

OS Images are built using components from the OpenHPC software stack. OpenHPC represents an aggregation of a number of common ingredients required to deploy and manage an HPC Linux* cluster including resource management, I/O clients, development tools, and a variety of scientific libraries. These packages have been pre-built with HPC integration in mind using a mix of open-source components. The documentation herein is intended to be reasonably generic,
but uses the underlying motivation of a small, 4-node state-full cluster installation to define a step-by-step process. 

Several optional customizations are included and the intent is that these collective instructions can be modified as needed for local site customizations.

 

\noindent {\bf Base Linux Edition}: this edition of the guide highlights
installation without the use of a companion configuration management system and
directly uses distro-provided package management tools for component
selection. The steps that follow also highlight specific changes to system
configuration files that are required as part of the cluster install
process. 

\input{common/audience}
\subsection{Requirements/Assumptions}

This installation recipe assumes the availability of an OpenStack controller (with neutron) node and four bare metal nodes. The Controller node serves as the central controller for OpenStack services and should have all the required OpenStack services installed and configured (i.e. keystone, nova, neutron, ironic along with their dependent services) to provision bare metal nodes with CentOS7.3 in a state-full 
configuration. 

This recipe is tested with OpenStack Mitaka release with CentOS 7.3. This 
document provides some examples for installing and configuring OpenStack using the Mitaka release of Packstack from RDO. More detail on using packstack can be found https://www.rdoproject.org/install/quickstart/. 

For power management, we assume that the bare metal node baseboard management controllers (BMCs) are available via IPMI from the chosen controller host. For file systems, we assume that the master server (instantiated during provisioning "HPC as a service") will host an NFS file system that is made available to the HPC compute nodes.



\begin{figure}[hbt]
\center
\includegraphics[width=0.85\linewidth]{HPCaaS-diagram.png}
\vspace*{-0.2cm}
\caption{Overview of physical cluster architecture.} \label{fig:physical_arch}
\end{figure}
\mbox{}

\vspace*{0.5cm}

An outline of the physical architecture discussed is shown in
Figure~\ref{fig:physical_arch} and highlights the high-level networking
configuration. The {\em master} host requires at least two Ethernet interfaces with {\em eth0} connected to the local data center network and {\em eth1} used to provision and manage the cluster backend (note that these interface names are examples and may be different depending on local settings and OS conventions). Two logical IP interfaces are expected to each compute node: the first is the standard Ethernet interface that will be used for provisioning and resource management. The second is used to connect to each host's BMC and is used for power management and remote console access. Physical connectivity for these two logical IP networks is often accommodated via separate cabling and switching infrastructure; however, an alternate configuration can also be accommodated via the use of a shared NIC, which runs a packet filter to divert management packets between the host and BMC.

 In addition to the IP networking, there is a high-speed network
(\InfiniBand{} in this recipe) that is also connected to each of the
hosts. This high speed network is used for application message passing and
optionally for \Lustre{} connectivity as well.

NOTE: This recipe sets various environment variables in one section and use them in other section. So users are expected to use single shell session for successful execution of this recipe. 

%Appendix A provides reference to pre-built recipe, and useful for users who wants to try out with minimum human interactions.


% -*- mode: latex; fill-column: 120; -*- 

\subsection{Inputs} \label{sec:inputs}
As this recipe details installing a cluster starting from bare-metal, there is a requirement to define IP addresses and  gather hardware MAC addresses in order to support a controlled provisioning process. These values are necessarily unique to the hardware being used, and this document uses variable substitution (\texttt{\$\{variable\}}) in the command-line examples that follow to highlight where local site inputs are required. 
A summary of the required and optional variables
used throughout this recipe are presented below. Note that while the example definitions above correspond to a small 4-node compute subsystem, the compute parameters are defined in array format to accommodate logical extension to larger node counts. 

\vspace*{0.2cm}
\begin{tabular}{@{}>{\textbullet}l p{7cm} l}
& \texttt{\$\{sms\_name\}} & {\small \# Hostname for SMS server} \\
& \texttt{\$\{sms\_ip\}} & {\small \# Internal IP address on SMS server}  \\
& \texttt{\$\{sms\_eth\_internal\}} & {\small \# Internal Ethernet interface on SMS} \\
& \texttt{\$\{sms\_mac\}} & {\small \# MAC address of baremetal machine to be provisioned as head node} \\
& \texttt{\$\{sms\_bmc\}} & {\small \# BMC IP address of baremetal machine to be provisioned as head node} \\
& \texttt{\$\{sms\_bmc\_username\}} & {\small \# BMC creadential of HPC head node instance} \\
& \texttt{\$\{sms\_bmc\_password\}} & {\small \# BMC creadential of HPC head node instance} \\
& \texttt{\$\{eth\_provision\}} & {\small \# Provisioning interface for computes} \\
& \texttt{\$\{internal\_netmask\}} & {\small \# Subnet netmask for internal network} \\
& \texttt{\$\{controller\_name\}} & {\small \# Host name for opensack controller node} \\
& \texttt{\$\{controller\_ip\}} & {\small \# internal ip address for opensack controller node} \\
& \texttt{\$\{cc\_subnet\_cidr\}} & {\small \# CIDR for cloud subnet, will be used to assign IP address to instances } \\
& \texttt{\$\{cc\_subnet\_dhcp\_start\}} & {\small \# Start IP address for cloud compute nodes instances} \\
& \texttt{\$\{cc\_subnet\_dhcp\_end\}} & {\small \# End IP address for cloud compute nodes instances} \\
& \texttt{\$\{ntp\_server\}} & {\small \# Local ntp server for time synchronization} \\
& \texttt{\$\{bmc\_username\}} & {\small \# BMC username for cloud computes, used by IPMI} \\
& \texttt{\$\{bmc\_password\}} & {\small \# BMC password for cloud computes, used by IPMI} \\
& \texttt{\$\{num\_computes\}} & {\small \# Total \# of desired cloud compute nodes} \\
& \texttt{\$\{compute\_regex\}} & {\small \# Regex matching all compute node names (e.g. ``c*'')} \\
& \texttt{\$\{cnodename\_prefix\}} & {\small \# Prefix for compute node names (e.g. ``c'')} \\
& \texttt{\$\{cc\_ip[0]\}}, \, \texttt{\$\{cc\_ip[1]\}}, ... & {\small \# Desired cloud compute node MAC addresses} \\
& \texttt{\$\{cc\_bmc[0]\}}, \texttt{\$\{cc\_bmc[1]\}}, ... & {\small \# BMC addresses for cloud compute nodes} \\
& \texttt{\$\{cc\_mac[0]\}}, \texttt{\$\{cc\_mac[1]\}}, ... & {\small \# MAC addresses for cloud computes nodes} \\
& \texttt{\$\{RAM\_MB\}}, & {\small \# Memory on each compute nodes} \\
& \texttt{\$\{CPU\_MB\}}, & {\small \# Number of CPUs on each compute nodes} \\
& \texttt{\$\{DISK\_GB\}}, & {\small \# Local sotrage on each compute nodes} \\
& \texttt{\$\{ARCH\}}, & {\small \# Processor Architecture of each compute nodes} \\
& \texttt{\$\{SOCKETS\}}, & {\small \# Number of sockets  on Compute. this is used by SLURM} \\
& \texttt{\$\{CORES\_PER\_SOCKET\}}, & {\small \# Number of cores per sockets  on Compute. This is used by SLURM} \\
& \texttt{\$\{THREADS\_PER\_CORE\}}, & {\small \# Number of threads per cores on Compute. This is used by SLURM} \\
& \texttt{\$\{chpc\_image\_deploy\_kernel\}}, & {\small \# Path to cloud node kernel image. } \\
& \texttt{\$\{chpc\_image\_deploy\_ramdisk\}}, & {\small \# Path to cloud node ramdisk image. } \\
& \texttt{\$\{chpc\_image\_user\}}, & {\small \# Path to HPC compute node user image. } \\
& \texttt{\$\{chpc\_image\_sms\}}, & {\small \# Path to HPC head node user image. } \\
& \texttt{\$\{ohpc\_pkg\}}, & {\small \# Link to OHPC package.  } \\
\end{tabular}
\begin{tabular}{@{}>{}l p{7cm} l}
& \texttt{} & {\small \ 	i.e. https://github.com/openhpc/ohpc/releases/download/ } \\
& \texttt{} & {\small \ 	v1.1.GA/ohpc\-release\-centos7.2\-1.1\-1.x86\_64.rpm. } \\
\end{tabular}

\vspace*{0.2cm}
\noindent {Optional:} 
\vspace*{0.1cm}

\begin{tabular}{@{}>{\textbullet}l p{7cm} l}
& \texttt{\$\{mgs\_fs\_name\}} & {\small \# Lustre MGS mount name} \\
& \texttt{\$\{sms\_ipoib\}} & {\small \# IPoIB address for SMS server} \\
& \texttt{\$\{ipoib\_netmask\}} & {\small \# Subnet netmask for internal IPoIB} \\
& \texttt{\$\{c\_ipoib[0]\}}, \texttt{\$\{c\_ipoib[1]\}}, ... & {\small \# IPoIB addresses for computes} \\
& \texttt{\$\{kargs\}} & {\small \# Kernel boot arguments} \\  
\end{tabular}




% Bare Metal Node Operating System --------------------------------------------
\clearpage
\section{Preparing Bare Metal Node Operating System}\label{sec:baremetalprep}
   This guide uses diskimage-builder utility to build and configure OS images 
for the sms (controller) node as well as compute nodes.  Preparing images is 
an optional part of the overall recipe. If user has predefined images, then 
environment variable "chpc\_create\_new\_image" must be reset and 
path to images must be provided using environment variable "chpc\_image\_deploy\_kernel", "chpc\_image\_deploy\_ramdisk",
"chpc\_image\_user", and "chpc\_image\_sms". 

In this example cloud images are built on controller node ("[ctrlr]\# "). Once images are built, they are stored in the standard openHPC Path.

\begin{lstlisting}[language=bash,keywords={}]
[ctrlr](*\#*) CHPC_CLOUD_IMAGE_PATH=/opt/ohpc/admin/images/cloud/
\end{lstlisting}


\subsection{Install and Setup diskimage-builder}\label{sec:dib_install}

	Images can be built on any supported OS. In this example we will install and build images on controller node, user can do same on a system independent of their production cluster. We will create common functions to install diskimage-builder and its dependencies from base OS distro.

	Create a function to install diskimage-builder

% begin_ohpc_run
% ohpc_validation_newline
%% ohpc_validation_comment # SECTION BMNOS
% ohpc_validation_comment function to install diskimage-builder
% ohpc_validation_comment
\begin{lstlisting}[language=bash,keywords={}]
[ctrlr](*\#*) function setup_dib() {
\end{lstlisting}
% end_ohpc_run
	Install diskimage-builder and its dependencies
% begin_ohpc_run
%% ohpc_validation_comment # SECTION BMNOS
% ohpc_validation_comment   Install diskimage-builder 
% ohpc_validation_comment
\begin{lstlisting}[language=bash,keywords={}]
[ctrlr](*\#*)     yum -y install diskimage-builder PyYAML

\end{lstlisting}
% end_ohpc_run

% begin_ohpc_run
%% ohpc_validation_comment # SECTION BMNOS
% ohpc_validation_comment   Install grub dependency 
% ohpc_validation_comment
\begin{lstlisting}[language=bash,keywords={}]
[ctrlr](*\#*)     yum -y install parted
\end{lstlisting}
% end_ohpc_run

	diskimage-builder installed from the base repository does not have a group install feature. So add our patch (Note: this probably will be included into RPM and will be part of that rpm installation)

% begin_ohpc_run
%% ohpc_validation_comment # SECTION BMNOS
% ohpc_validation_comment   Fix for group install
% ohpc_validation_comment

\begin{lstlisting}[language=bash,keywords={}]
[ctrlr](*\#*)     yum -y install <DIB patch>
[ctrlr](*\#*) } # end of function
\end{lstlisting}
% end_ohpc_run


\subsection{Setup common environment for diskimage-builder}\label{sec:dib_environment}
diskimage-builder or dib uses environment variables and elements to customize the images. For debugging purpose, we will create default user chpc with a password intel8086, with sudo privilege. These variables are used by element devuser. 

% begin_ohpc_run
% ohpc_validation_newline
% ohpc_validation_comment # SECTION BMNOS
% ohpc_validation_comment # diskimage-builder initial config
% ohpc_validation_comment
\begin{lstlisting}[language=bash,keywords={}]
[ctrlr](*\#*) export DIB_DEV_USER_USERNAME=chpc
[ctrlr](*\#*) export DIB_DEV_USER_PASSWORD=intel8086
[ctrlr](*\#*) export DIB_DEV_USER_PWDLESS_SUDO=1
\end{lstlisting}
% end_phpc_run

Now add path to custom elements which are not part of base diskimage-builder. OpenHPC provides few HPC elements. [Note: This also can be part of openHPC provided rpm for dib. In that case remove this step]

% begin_ohpc_run
% ohpc_validation_newline
% ohpc_validation_comment # SECTION BMNOS
% ohpc_validation_comment # Add custom elements in DIB
% ohpc_validation_comment
\begin{lstlisting}[language=bash,keywords={}]
[ctrlr](*\#*) export ELEMENTS_PATH="$(realpath ../../dib/hpc/elements)"
\end{lstlisting}

Add path to HPC specific files [note: same as earlier, this too can be part of rpm package]

% begin_ohpc_run
% ohpc_validation_newline
% ohpc_validation_comment # SECTION BMNOS
% ohpc_validation_comment # Add path to HPC specific files
% ohpc_validation_comment
\begin{lstlisting}[language=bash,keywords={}]
[ctrlr](*\#*) export DIB_HPC_FILE_PATH="$(realpath ../../dib/hpc/hpc-files/)"
\end{lstlisting}

HPC elements are common for OpenHPC and Intel HPC Orchestrator, environment variable "DIB\_HPC\_BASE" tell dib which one to pick. For OpenHPC set environment variable

% begin_ohpc_run
% ohpc_validation_newline
% ohpc_validation_comment # SECTION BMNOS
% ohpc_validation_comment # Set DIB Element
% ohpc_validation_comment
\begin{lstlisting}[language=bash,keywords={}]
[ctrlr](*\#*) export DIB_ HPC_BASE="ohpc"
\end{lstlisting}
% end_ohpc_run

Make sure open hpc packages is installed. ohpc\_pkg is one of the input setup earlier in this document.

% begin_ohpc_run
% ohpc_validation_newline
% ohpc_validation_comment # SECTION BMNOS
% ohpc_validation_comment # Check package installed
% ohpc_validation_comment
\begin{lstlisting}[language=bash,keywords={}]
[ctrlr](*\#*) yum -y install ${ohpc_pkg}
\end{lstlisting}
% end_ohpc_run

Export same to DIB.

% begin_ohpc_run
% ohpc_validation_newline
% ohpc_validation_comment # SECTION BMNOS
% ohpc_validation_comment # Export to DIB
% ohpc_validation_comment
\begin{lstlisting}[language=bash,keywords={}]
[ctrlr](*\#*) export DIB_HPC_OHPC_PKG=${ohpc_pkg}
\end{lstlisting}
% end_ohpc_run

Create list of HPC elements needed to build HPC images, by starting hpc-env-base. This element will setup basic hpc environment to build hpc images.

% begin_ohpc_run
% ohpc_validation_newline
% ohpc_validation_comment # SECTION BMNOS
% ohpc_validation_comment # Create listof elements
% ohpc_validation_comment
\begin{lstlisting}[language=bash,keywords={}]
[ctrlr](*\#*) DIB_HPC_ELEMENTS="hpc-env-base"
\end{lstlisting}
% end_ohpc_run

\subsection{Preparing ironic deploy images}\label{sec:ironic_deploy_images}
Ironic uses deploy images (aka kernel) to bootstrap the provisioning of user images. 
Unset any previous environment flag

% begin_ohpc_run
% ohpc_validation_newline

\begin{lstlisting}[language=bash,keywords={}]
[ctrlr](*\#*) DIB_YUM_REPO_CONF
\end{lstlisting} 
 % end_ohpc_run 


Git is used by some of the elements in diskimage-builder. 
% begin_ohpc_run
% ohpc_validation_newline

\begin{lstlisting}[language=bash,keywords={}]
[ctrlr](*\#*) yum -y install git
\end{lstlisting} 
 % end_ohpc_run 


Create deploy images using disk-image-create cli. This will download base centos image from distro, install ironic-agent on it and create kernel and initiramfs images.
% begin_ohpc_run
% ohpc_validation_newline

\begin{lstlisting}[language=bash,keywords={}]
[ctrlr](*\#*) disk-image-create ironic-agent centos7 -o icloud-hpc-deploy-c7
\end{lstlisting} 
 % end_ohpc_run

Move deploy images to ohpc standard path.
% begin_ohpc_run
% ohpc_validation_newline

\begin{lstlisting}[language=bash,keywords={}]
[ctrlr](*\#*) chpc_image_deploy_kernel="$( realpath icloud-hpc-deploy-c7.kernel)"
[ctrlr](*\#*) chpc_image_deploy_ramdisk="$( realpath icloud-hpc-deploy-c7.initramfs)"
[ctrlr](*\#*) # Store Images file
[ctrlr](*\#*) mkdir -p $CHPC_CLOUD_IMAGE_PATH/
[ctrlr](*\#*) sudo mv -f $chpc_image_deploy_kernel $CHPC_CLOUD_IMAGE_PATH/
[ctrlr](*\#*) chpc_image_deploy_kernel=$CHPC_CLOUD_IMAGE_PATH/$(basename $chpc_image_deploy_kernel)
[ctrlr](*\#*) sudo mv -f $chpc_image_deploy_ramdisk $CHPC_CLOUD_IMAGE_PATH/
[ctrlr](*\#*) chpc_image_deploy_ramdisk=$CHPC_CLOUD_IMAGE_PATH/$(basename $chpc_image_deploy_ramdisk)
\end{lstlisting} 
 % end_ohpc_run


\subsection{Preparing user images for bare metal instances}\label{sec:bare_metal_user_images}
For "HPC as a Service" we will building two user images (one for sms node and one for compute node) and two deploy images. User images we build here will be customized with OpenHPC using HPC specific elements. 

\subsubsection{Preparing user image for head node OS}\label{sec:head_node_images}
	Building head node (SMS) images requires installing the server packages of HPC components into the image. This is accomplished by setting image type to sms. Default image type in hpc elements is "compute". We will create a function prepare\_sms\_image to build this image, which will be called later in document.

% begin_ohpc_run
% ohpc_validation_newline
% ohpc_validation_comment function to prepare head node
% ohpc_validation_comment
\begin{lstlisting}[language=bash,keywords={}]
[ctrlr](*\#*) function prepare_sms_image() {
\end{lstlisting} 
% end_ohpc_run

	Check if user has requested to create images by verifying environment variables chpc\_create\_new\_image and chpc\_image\_sms. If user already supplied images then we will copy images to our common image location and setup environment variable for later use.
	
% begin_ohpc_run
\begin{lstlisting}[language=bash,keywords={}]
[ctrlr](*\#*)     if [[ ${chpc_create_new_image} -ne 1 ]] && [[ -s $chpc_image_sms ]]; then
[ctrlr](*\#*)         # No need to create an image, image is provided by user
[ctrlr](*\#*)         echo -n "Skiping cloud sms-image build, Image provided:"
[ctrlr](*\#*)         echo "$chpc_image_sms"
[ctrlr](*\#*)         CHPC_IMAGE_DEST=$CHPC_CLOUD_IMAGE_PATH/$(basename $chpc_image_sms)
[ctrlr](*\#*)         if [[ ! -e $CHPC_IMAGE_DEST ]]; then
[ctrlr](*\#*)             sudo cp $chpc_image_sms $CHPC_CLOUD_IMAGE_PATH
[ctrlr](*\#*)         fi
[ctrlr](*\#*)         chpc_image_sms=$CHPC_IMAGE_DEST
[ctrlr](*\#*)     else
\end{lstlisting} 
% end_ohpc_run

	If user has not supplied images then we will build the sms image here. Disk-image-builder supports two types of HPC images, "sms" and "compute".  

% begin_ohpc_run
\begin{lstlisting}[language=bash,keywords={}]
[ctrlr](*\#*)         # setup environment varioable to indicate sms image type
[ctrlr](*\#*)         setup_dib_hpc_base
[ctrlr](*\#*)         export DIB_HPC_IMAGE_TYPE=sms
\end{lstlisting} 
% end_ohpc_run

	Enable SLURM resource manager for "head" node.

% begin_ohpc_run

\begin{lstlisting}[language=bash,keywords={}]
[ctrlr](*\#*)         DIB_HPC_ELEMENTS+=" hpc-slurm"
\end{lstlisting} 
 % end_ohpc_run

	Add optional OpenHPC components.

% begin_ohpc_run

\begin{lstlisting}[language=bash,keywords={}]
[ctrlr](*\#*)         if [[ ${enable_mrsh} -eq 1 ]];then
[ctrlr](*\#*)             DIB_HPC_ELEMENTS+=" hpc-mrsh"
[ctrlr](*\#*)         fi
\end{lstlisting} 
 % end_ohpc_run

	We will also setup an HPC development environment on the HPC head node. 

	Start by enabling gnu compiler on head node.

% begin_ohpc_run

\begin{lstlisting}[language=bash,keywords={}]
[ctrlr](*\#*)         export DIB_HPC_COMPILER="gnu"
\end{lstlisting} 
 % end_ohpc_run

	Enable openmpi \& mvapich2.

% begin_ohpc_run

\begin{lstlisting}[language=bash,keywords={}]
[ctrlr](*\#*)         export DIB_HPC_MPI="openmpi mvapich2"
\end{lstlisting} 
 % end_ohpc_run

	Enable performance tools.

% begin_ohpc_run

\begin{lstlisting}[language=bash,keywords={}]
[ctrlr](*\#*)         export DIB_HPC_PERF_TOOLS="perf-tools"
\end{lstlisting} 
 % end_ohpc_run

	Enable 3rd party libraries serial-libs, parallel-libs, io-libs, python-libs and runtimes.

% begin_ohpc_run

\begin{lstlisting}[language=bash,keywords={}]
[ctrlr](*\#*)         export DIB_HPC_3RD_LIBS="serial-libs parallel-libs io-libs python-libs runtimes"
\end{lstlisting} 
 % end_ohpc_run

	Add HPC development environment element to list of elements.

% begin_ohpc_run

\begin{lstlisting}[language=bash,keywords={}]
[ctrlr](*\#*)         DIB_HPC_ELEMENTS+=" hpc-dev-env"
\end{lstlisting} 
 % end_ohpc_run

	Now create an sms image with element local-config, dhcp-all-interfaces, devuser, selinux-permisive, and all HPC specific elements. Element local-config copies your local environment into image, which is the local users, their password and permissions. Element devuser will create a new user specified by environment variable "DIB\_DEV\_USER\_USERNAME". The image will be named  {\em  icloud-hpc-cent7-sms }.

% begin_ohpc_run

\begin{lstlisting}[language=bash,keywords={}]
[ctrlr](*\#*)         disk-image-create centos7 vm local-config dhcp-all-interfaces \
[ctrlr](*\#*)         	devuser selinux-permissive $DIB_HPC_ELEMENTS -o icloud-hpc-cent7-sms
\end{lstlisting} 
 % end_ohpc_run

	It will take a while to build an image. Once the image is built, copy it to the standard OpenHPC path.

% begin_ohpc_run

\begin{lstlisting}[language=bash,keywords={}]
[ctrlr](*\#*)         chpc_image_sms="$( realpath icloud-hpc-cent7-sms.qcow2)"
[ctrlr](*\#*)         mkdir -p $CHPC_CLOUD_IMAGE_PATH
[ctrlr](*\#*)         mv -f $chpc_image_sms $CHPC_CLOUD_IMAGE_PATH
[ctrlr](*\#*)         chpc_image_sms=$CHPC_CLOUD_IMAGE_PATH/$(basename $chpc_image_sms)
[ctrlr](*\#*)     fi # end of else of or if
[ctrlr](*\#*) } # end of function
\end{lstlisting} 
 % end_ohpc_run


\subsubsection{Preparing user image for compute node OS}\label{sec:compute_node_images}
To build compute node images, we need to install client packages of HPC components. This is accomplished by setting image type to compute. Default image type in hpc elements is "compute".

\begin{lstlisting}[language=bash,keywords={}]
[ctrlr](*\#*) export DIB_HPC_IMAGE_TYPE=compute
\end{lstlisting}

Now enable SLURM resource manager for compute node.

\begin{lstlisting}[language=bash,keywords={}]
[ctrlr](*\#*) DIB_HPC_ELEMENTS+=" hpc-slurm"
\end{lstlisting}

Add optional OpenHPC Components

\begin{lstlisting}[language=bash,keywords={}]
[ctrlr](*\#*)  if [[ ${enable_mrsh} -eq 1 ]];then
[ctrlr](*\#*)         DIB_HPC_ELEMENTS+=" hpc-mrsh"
[ctrlr](*\#*)  fi
\end{lstlisting}

Now create a compute node image with element local-config, dhcp-all-interfaces, devuser, selinux-permisive and all hpc specific elements. Element local-config copies your local environment into image, which is the local users, their password and permissions. Element devuser will create new user specified by environment variable "DIB\_DEV\_USER\_USERNAME". 

\begin{lstlisting}[language=bash,keywords={}]
[ctrlr](*\#*) disk-image-create centos7 vm local-config dhcp-all-interfaces devuser \
 selinux-permissive $DIB_HPC_ELEMENTS -o icloud-hpc-cent7-sms
\end{lstlisting}

It will take a while to build an image. Once image is built, copy it to standard OpenHPC path.

\begin{lstlisting}[language=bash,keywords={}]
[ctrlr](*\#*) chpc_image_sms="$( realpath icloud-hpc-cent7.qcow2)"
[ctrlr](*\#*) mkdir -p $CHPC_CLOUD_IMAGE_PATH
[ctrlr](*\#*) mv -f $chpc_image_user$CHPC_CLOUD_IMAGE_PATH
[ctrlr](*\#*) chpc_image_user=$CHPC_CLOUD_IMAGE_PATH/$(basename $chpc_image_sms)
\end{lstlisting}



\subsection{Introduction to diskimage-builder}\label{sec:din_intro}
It is a utility to build and configure OS images for sms node as well compute node. It uses prebuild minimum OS images from base distro, which it further customizes as per user request. Diskimage-builder is a framework which uses many elements (similar to plug-ins) to customize the image. Base distribution of diskimage-builder comes with pre-defined elements. This recipe uses additional HPC elements which were built to customize images based on OpenHPC components.

\subsubsection{HPC Elements to build OpenHPC Images}

"HPC as a service" uses 4 HPC specific elements in addition to pre-packaged elements comes with diskimage-builder.

hpc-dev-env: 
	This is mainly used to create sms images to create HPC development environment. It creates hpc development environment by installing following OpenHPC components within image: 
		ohpc-autotools, valgrind-ohpc, easybuild-ohpc, spack-ohpc, r\_base-ohpc
		mpi and compiler for chosen MPI and compiler via environment variable, \$DIB\_HPC\_COMPILER, \$DIB\_HPC\_MPI
		Performance Tools lmod-default with their 3rd party libraries.

hpc-env-base: TBC

hpc-mrsh:TBC

hpc-slurm: TBC

\subsubsection{Editing HPC Elements}

	TBC



% ------------------------------------------------------------------
\clearpage

\section{Preparing Cloud-Init} \label{sec:cloud-init_prep}
	OpenStack uses cloud-init for boot time initialization of cloud instances. This recipe relies on cloud-init to initialize HPC instances in an OpenStack cloud. This recipe prepares the cloud-init initialization template script, which is then updated with sms-ip and other environment variables just before the provisioning. This is than fed as user data to Nova during instance creation. The script generated here will be executed by root during bootup. 
The cloud-init templates are provided as a part "docs-chpc" RPM. Here we will create a similar template and store this template into the same path as installed by RPM. 



\subsection{Preparing template for compute node cloud-init} \label{sec:c_i-template_compute_node}
	Create an empty chpc\_init file and open it for editing. You can also modify the existing template. 

	Start editing the new file by adding environment variables. First, set the path to the shared folder for cloud-init and then declere variable sms\_name with "update\_sms\_name", which we will update just before provisioning.

% begin_ohpc_run
% ohpc_validation_newline
% ohpc_validation_comment #   XFILEX
% ohpc_validation_comment #   PFILEP
% ohpc_command #!/bin/bash
% ohpc_validation_comment #   FILE: chpc_init

\begin{lstlisting}[language=bash,keywords={}]
[ctrlr](*\#*) #Ensure the executing shell is in the same directory as the script.
[ctrlr](*\#*) SCRIPTDIR="$( cd "$( dirname "$( readlink -f "${BASH_SOURCE[0]}" )" )" && pwd -P && echo x)"
[ctrlr](*\#*) SCRIPTDIR="${SCRIPTDIR%x}"
[ctrlr](*\#*) cd $SCRIPTDIR
[ctrlr](*\#*) chpcInitPath=/opt/ohpc/admin/cloud_hpc_init
[ctrlr](*\#*) sms_name=<update_sms_name>

[ctrlr](*\#*) logger "chpcInit: Updating Compute Node with HPC configuration"
\end{lstlisting}
% end_ohpc_run

	Update the rsyslog configuration file to send the syslog to sms. "sms\_ip" is the tag used, which is updated with the IP address of the SMS node just before provisioning.

% begin_ohpc_run
% ohpc_validation_newline
\begin{lstlisting}[language=bash,keywords={}]
[ctrlr](*\#*) # Update rsyslog
[ctrlr](*\#*) cat /etc/rsyslog.conf | grep "<sms_ip>:514"
[ctrlr](*\#*) rsyslog_set=$?
[ctrlr](*\#*) if [ "${rsyslog_set}" -ne "0" ]; then
[ctrlr](*\#*)    echo "*.* @<sms_ip>:514" >> /etc/rsyslog.conf
[ctrlr](*\#*) fi
[ctrlr](*\#*) systemctl restart rsyslog
[ctrlr](*\#*) logger "chpcInit: rsyslog configuration complete, updating remaining HPC configuration"
\end{lstlisting}
% end_ohpc_run

	Assuming sms node share the directories "/home", "/opt/ohpc/pub", and "/opt/ohpc/admin/cloud\_hpc\_init" via nfs, we will mount them on compute nodes during boot.

% begin_ohpc_run
% ohpc_validation_newline
\begin{lstlisting}[language=bash,keywords={}]
[ctrlr](*\#*) # nfs mount directory from SMS head node to Compute Node
[ctrlr](*\#*) cat /etc/fstab | grep "<sms_ip>:/home"
[ctrlr](*\#*) home_exists=$?
[ctrlr](*\#*) if [ "${home_exists}" -ne "0" ]; then
[ctrlr](*\#*)     echo "<sms_ip>:/home /home nfs nfsvers=3,rsize=1024,wsize=1024,cto 0 0" >> /etc/fstab
[ctrlr](*\#*) fi
[ctrlr](*\#*) cat /etc/fstab | grep "<sms_ip>:/opt/ohpc/pub"
[ctrlr](*\#*) ohpc_pub_exists=$?
[ctrlr](*\#*) 
[ctrlr](*\#*) if [ "${ohpc_pub_exists}" -ne "0" ]; then
[ctrlr](*\#*)     echo "<sms_ip>:/opt/ohpc/pub /opt/ohpc/pub nfs nfsvers=3 0 0" >> /etc/fstab
[ctrlr](*\#*)     # Make sure we have directory to mount
[ctrlr](*\#*)     # Clean up if required
[ctrlr](*\#*)     if [ -e /opt/ohpc/pub ]; then
[ctrlr](*\#*)         echo "chpcInit: [WARNING] /opt/ohpc/pub already exists!!"
[ctrlr](*\#*)     fi
[ctrlr](*\#*) fi
[ctrlr](*\#*) mkdir -p /opt/ohpc/pub
[ctrlr](*\#*) mount /home
[ctrlr](*\#*) mount /opt/ohpc/pub
[ctrlr](*\#*) 
[ctrlr](*\#*) # Mount cloud_hpc_init
[ctrlr](*\#*) cat /etc/fstab | grep "sms_ip:$chpcInitPath"
[ctrlr](*\#*) CloudHPCInit_exist=$?
[ctrlr](*\#*) if [ "${CloudHPCInit_exist}" -ne "0" ]; then
[ctrlr](*\#*)     echo "<sms_ip>:$chpcInitPath $chpcInitPath nfs nfsvers=3 0 0" >> /etc/fstab
[ctrlr](*\#*) fi
[ctrlr](*\#*) mkdir -p $chpcInitPath
[ctrlr](*\#*) mount $chpcInitPath
[ctrlr](*\#*) # Restart nfs
[ctrlr](*\#*) systemctl restart nfs
[ctrlr](*\#*) 
\end{lstlisting}
% end_ohpc_run

	Have ntp sync with the sms node and then copy shared keys.

% begin_ohpc_run
% ohpc_validation_newline

\begin{lstlisting}
[ctrlr](*\#*) # Restart ntp at the compute node
[ctrlr](*\#*) systemctl enable ntpd
[ctrlr](*\#*) # Update ntp server
[ctrlr](*\#*) cat /etc/ntp.conf | grep "server <sms_ip>"
[ctrlr](*\#*) ntp_server_exists=$?
[ctrlr](*\#*) if [ "${ntp_server_exists}" -ne "0" ]; then
[ctrlr](*\#*)     echo "server <sms_ip>" >> /etc/ntp.conf
[ctrlr](*\#*) fi
[ctrlr](*\#*) systemctl restart ntpd
[ctrlr](*\#*) # Sync time
[ctrlr](*\#*) ntpstat
[ctrlr](*\#*) if [ -d $chpcInitPath ]; then
[ctrlr](*\#*)     # Copy public keys
[ctrlr](*\#*)     cp -f -L $chpcInitPath/authorized_keys /root/.ssh/
[ctrlr](*\#*) else
[ctrlr](*\#*)     logger "chpcInit:ERROR: cannot stat nfs shared /opt directory, cannot copy HPC system files"
[ctrlr](*\#*) fi
[ctrlr](*\#*) #Change file permissions in /etc/ssh to fix ssh into compute node
[ctrlr](*\#*) chmod 0600 /etc/ssh/ssh_host_*_key
\end{lstlisting}
% end_ohpc_run

% begin_ohpc_run
% ohpc_validation_newline




%%% DO I STILL NEED THIS ??? %%%
%%%In addition to the \OHPC{} package repository, the {\em master} host also
%%%requires access to the standard base OS distro repositories in order to resolve
%%%necessary dependencies. For \baseOS{}, the requirements are to have access to
%%%both the base OS and EPEL repositories for which mirrors are freely available online:
%%%
%%%\begin{itemize*}
%%%\item CentOS-7 - Base 7.3.1611
%%%  (e.g. \href{http://mirror.centos.org/centos-7/7/os/x86\_64}
%%%             {\color{blue}{http://mirror.centos.org/centos-7/7/os/x86\_64}} )
%%%\item EPEL 7 (e.g. \href{http://download.fedoraproject.org/pub/epel/7/x86\_64}
%%%                       {\color{blue}{http://download.fedoraproject.org/pub/epel/7/x86\_64}} )
%%%\end{itemize*}
%%%
%%%\noindent The public EPEL repository will be enabled automatically upon installation of the 
%%%\texttt{ohpc-release} package. Note that this requires the CentOS Extras
%%%repository, which is shipped with CentOS and is enabled by default.



\subsection{Preparing template for sms node cloud-init} \label{sec:c_i-template-sms-node}
	The cloud-init script for sms node is little different than compute node. The SMS node, when instantiated within Openstack, serves as a head node for HPC-as-a-Service, and hosts all the services as a sms node in an independent HPC cluster. This will host the server side of applications, resource manager, and share users. For more detail on sms node functionality please refer to OpenHPC documentation.

	In this recipe, we will prepare cloud-init template script for sms node, which then is updated with compute node IP, NTP server, and other environmental variables, just before provisioning. 

	Create an empty chpc\_init file and open for editing. You can also use existing template and modify. Start editing by adding some environment variable, which will be updated later, just before provisioning.

% begin_ohpc_run
% ohpc_validation_newline
% ohpc_validation_comment #   PFILEP

% ohpc_validation_comment # FILE: chpc_sms_init

\begin{lstlisting}[language=bash,keywords={}]
[ctrlr](*\#*) # Get the Compute node prefix and number of compute nodes
[ctrlr](*\#*) cnodename_prefix=<update_cnodename_prefix>
[ctrlr](*\#*) num_ccomputes=<update_num_ccomputes>
[ctrlr](*\#*) ntp_server=<update_ntp_server>
[ctrlr](*\#*) sms_name=<update_sms_name>
[ctrlr](*\#*) logger "chpcInit: Entered chpcInit"
[ctrlr](*\#*) #Ensure the executing shell is in the same directory as the script.
[ctrlr](*\#*) SCRIPTDIR="$( cd "$( dirname "$( readlink -f "${BASH_SOURCE[0]}" )" )" && pwd -P && echo x)"
[ctrlr](*\#*) SCRIPTDIR="${SCRIPTDIR%x}"
[ctrlr](*\#*) cd $SCRIPTDIR

\end{lstlisting} 
 % end_ohpc_run

	Now, setup nfs share for cloud-init and files which want to send to compute nodes.

% begin_ohpc_run
% ohpc_validation_newline

\begin{lstlisting}[language=bash,keywords={}]
[ctrlr](*\#*) # setup cloudinit directory
[ctrlr](*\#*) chpcInitPath=/opt/ohpc/admin/cloud_hpc_init
[ctrlr](*\#*) # create directory of not exists
[ctrlr](*\#*) mkdir -p $chpcInitPath
[ctrlr](*\#*) chmod 700 $chpcInitPath
[ctrlr](*\#*) # To create same user environment, copy user files 
[ctrlr](*\#*) # Copy other files needed for Cloud Init
[ctrlr](*\#*) sudo cp -fpr /etc/passwd $chpcInitPath
[ctrlr](*\#*) sudo cp -fpr /etc/shadow $chpcInitPath
[ctrlr](*\#*) sudo cp -fpr /etc/group $chpcInitPath
[ctrlr](*\#*) # Copy public ssh key to shared drive
[ctrlr](*\#*) _ssh_path=/root/.ssh
[ctrlr](*\#*) if [ ! -e "$_ssh_path/hpcasservice" ]; then
[ctrlr](*\#*) 
[ctrlr](*\#*) 	if [ ! -d "$_ssh_path" ]; then
[ctrlr](*\#*) 		install -d -m 700 $_ssh_path
[ctrlr](*\#*) 	fi
[ctrlr](*\#*) 	ssh-keygen -t dsa -f $_ssh_path/hpcasservice -N '' -C "HPC Cluster key" > /dev/null 2>&1
[ctrlr](*\#*) 	cat $_ssh_path/hpcasservice.pub >> $_ssh_path/authorized_keys
[ctrlr](*\#*) 	chmod 0600 $_ssh_path/authorized_keys
[ctrlr](*\#*) fi
[ctrlr](*\#*) 	#update config
[ctrlr](*\#*) if [ ! -e "$_ssh_path/config" ]; then
[ctrlr](*\#*) 	echo "Host *" > $_ssh_path/config
[ctrlr](*\#*) 	echo "    IdentityFile ~/.ssh/hpcasservice" >> $_ssh_path/config
[ctrlr](*\#*) 	echo "    StrictHostKeyChecking=no" >> $_ssh_path/config
[ctrlr](*\#*) fi
[ctrlr](*\#*) cp -fpr $_ssh_path/authorized_keys $chpcInitPath

\end{lstlisting} 
 % end_ohpc_run

	Share /home, /opt/ohpc/pub and /opt/ohpc/admin/cloud\_hpc\_init over nfs

% begin_ohpc_run
% ohpc_validation_newline

\begin{lstlisting}[language=bash,keywords={}]
[ctrlr](*\#*) # export CloudInit Path to nfs share
[ctrlr](*\#*) cat /etc/exports | grep "$chpcInitPath"
[ctrlr](*\#*) chpcInitPath_exported=$?
[ctrlr](*\#*) 
[ctrlr](*\#*) if [ "${chpcInitPath_exported}" -ne "0" ]; then
[ctrlr](*\#*)    echo "$chpcInitPath *(rw,no_subtree_check,no_root_squash)" >> /etc/exports
[ctrlr](*\#*) fi
[ctrlr](*\#*) # share /home from HN
[ctrlr](*\#*) if ! grep "^/home" /etc/exports; then
[ctrlr](*\#*)     echo "/home *(rw,no_subtree_check,fsid=10,no_root_squash)" >> /etc/exports
[ctrlr](*\#*) fi
[ctrlr](*\#*) # share /opt/ from HN
[ctrlr](*\#*) if ! grep "^/opt/ohpc/pub" /etc/exports; then
[ctrlr](*\#*)     echo "/opt/ohpc/pub *(ro,no_subtree_check,fsid=11)" >> /etc/exports
[ctrlr](*\#*) fi
[ctrlr](*\#*) exportfs -a
[ctrlr](*\#*) # Restart nfs
[ctrlr](*\#*) systemctl restart nfs
[ctrlr](*\#*) systemctl enable nfs-server
[ctrlr](*\#*) logger "chpcInit: nfs configuration complete, updating remaining HPC configuration" 
\end{lstlisting} 
 % end_ohpc_run

	Configure ntp sever on sms node, as per the site setting.

% begin_ohpc_run
% ohpc_validation_newline

\begin{lstlisting}[language=bash,keywords={}]
[ctrlr](*\#*) # configure NTP
[ctrlr](*\#*) systemctl enable ntpd
[ctrlr](*\#*) if [[ ! -z "$ntp_server" ]]; then
[ctrlr](*\#*)    echo "server ${ntp_server}" >> /etc/ntp.conf
[ctrlr](*\#*) fi
[ctrlr](*\#*) systemctl restart ntpd
[ctrlr](*\#*) systemctl enable ntpd.service
[ctrlr](*\#*) # time sync
[ctrlr](*\#*) ntpstat
[ctrlr](*\#*) logger "chpcInit:ntp configuration done"
\end{lstlisting} 
 % end_ohpc_run

	Distribute munge keys with compute nodes, and then update the SLURM resource manager with HPC compute nodes.


% begin_ohpc_run
% ohpc_validation_newline

\begin{lstlisting}[language=bash,keywords={}]
[ctrlr](*\#*) #(*\#*)(*\#*) Update Resource manager configuration (*\#*)(*\#*)(*\#*)
[ctrlr](*\#*) # Update basic slurm configuration at sms node
[ctrlr](*\#*) perl -pi -e "s/ControlMachine=\S+/ControlMachine=${sms_name}/" /etc/slurm/slurm.conf
[ctrlr](*\#*) perl -pi -e "s/^NodeName=(\S+)/NodeName=${cnodename_prefix}[1-${num_ccomputes}]/" /etc/slurm/slurm.conf
[ctrlr](*\#*) perl -pi -e "s/^PartitionName=normal Nodes=(\S+)/PartitionName=normal Nodes=${cnodename_prefix}[1-${num_ccomputes}]/" /etc/slurm/slurm.conf
[ctrlr](*\#*) # copy slurm file from sms node to Cloud Comute Nodes
[ctrlr](*\#*) cp -fpr -L /etc/slurm/slurm.conf $chpcInitPath
[ctrlr](*\#*) cp -fpr -L /etc/pam.d/slurm $chpcInitPath
[ctrlr](*\#*) cp -fpr -L /etc/munge/munge.key $chpcInitPath
[ctrlr](*\#*) # Start slurm and munge 
[ctrlr](*\#*) systemctl enable munge
[ctrlr](*\#*) systemctl restart munge
[ctrlr](*\#*) systemctl enable slurmctld
[ctrlr](*\#*) systemctl restart slurmctld
[ctrlr](*\#*) #systemctl enable slurmd
[ctrlr](*\#*) #systemctl restart slurmd
[ctrlr](*\#*) logger "chpcInit:slurm configuration done"
\end{lstlisting} 
 % end_ohpc_run


	One last step is to make sure ssh is working and enabled on compute nodes. Update/verify the permissions of ssh.

% begin_ohpc_run
% ohpc_validation_newline

\begin{lstlisting}[language=bash,keywords={}]
[ctrlr](*\#*) #Change file permissions in /etc/ssh to fix ssh into compute node
[ctrlr](*\#*) chmod 0600 /etc/ssh/ssh_host_*_key
\end{lstlisting} 
 % end_ohpc_run

	Save the file with name chp\_sms\_cinit, we will use this file during sms node instance creation.


	

\subsection{Prepare optional part of cloud-init} \label{sec:c_i-optional}

\subsubsection{Update mrsh during cloud-init}
 
Create a new file sms/update\_mrsh and add mrsh configuration to enable mrsh on sms node. And save it. 

% begin_ohpc_run
% ohpc_validation_newline

\begin{lstlisting}[language=bash,keywords={}]
[ctrlr](*\#*) # Update mrsh
[ctrlr](*\#*) # check if it is already configured grep mshell /etc/services will return non-zero, else configure"
[ctrlr](*\#*) cat /etc/services | grep mshell
[ctrlr](*\#*) mshell_exists=$?
[ctrlr](*\#*) if [ "${mshell_exists}" -ne "0" ]; then
[ctrlr](*\#*)     echo "mshell          21212/tcp                  (*\#*) mrshd" >> /etc/services
[ctrlr](*\#*) fi
[ctrlr](*\#*) cat /etc/services | grep mlogin
[ctrlr](*\#*) mlogin_exists=$?
[ctrlr](*\#*) if [ "${mlogin_exists}" -ne "0" ]; then
[ctrlr](*\#*)     echo "mlogin            541/tcp                  (*\#*) mrlogind" >> /etc/services
[ctrlr](*\#*) fi
\end{lstlisting} 
 % end_ohpc_run


\subsubsection{Updating cluster shell during cloud -init}
Create a new file sms/update\_clustershell and add configuration to enable clustershell on sms node. And save it. 

% begin_ohpc_run
% ohpc_validation_newline


\begin{lstlisting}[language=bash,keywords={}]
[ctrlr](*\#*) sed -i -- 's/all: @adm,@compute/compute: cc[1-${num_ccomputes}]\n&/' /etc/clustershell/groups.d/local.cfg
\end{lstlisting} 
 % end_ohpc_run

\subsection{Configuring overall cloud-init} \label{sec:c_i-config}

In previous section we created template for cloud-init for hpc head node and hpc compute nodes. We need to update these template with user defined inputs like IP Address, node names. With these updates, cloud-init script is ready to deploy with OpenStack Nova.

Copy cloud-init template to working folder

\begin{lstlisting}[language=bash,keywords={}]
chpcInitPath=/opt/ohpc/admin/cloud_hpc_init
# if directory exists then mv to Old directory. TBD
mkdir -p $chpcInitPath
#copy Cloud HPC files to temp working directory
sudo cp -fr -L < ${SCRIPTDIR} >/ cloud_hpc_init/${chpc_base}/* $chpcInitPath/
export chpcInit=$chpcInitPath/chpc_init
export chpcSMSInit=$chpcInitPath/chpc_sms_init
\end{lstlisting}


Update sms\_ip in compute node cloud-init template with HPC head node. 

\begin{lstlisting}[language=bash,keywords={}]
sudo sed -i -e "s/<sms_ip>/${sms_ip}/g" $chpcInit
\end{lstlisting}


Update HPC head node cloud-init template with compute name prefix as defined by user

\begin{lstlisting}[language=bash,keywords={}]
sudo sed -i -e "s/<update_cnodename_prefix>/${cnodename_prefix}/g" $chpcSMSInit
sudo sed -i -e "s/<update_num_ccomputes>/${num_ccomputes}/g" $chpcSMSInit
Update hostname of HPC head node & NTP server information
sudo sed -i -e "s/<update_ntp_server>/${controller_ip}/g" $chpcSMSInit
sudo sed -i -e "s/<update_sms_name>/${sms_name}/g" $chpcSMSInit
\end{lstlisting}

Optionally if user enabled mrsh or clusteshell, then update cloud-init accordingly

\begin{lstlisting}[language=bash,keywords={}]
if [[ ${enable_mrsh} -eq 1 ]];then
   # update mrsh for sms node
   cat $CHPC_SCRIPTDIR/sms/update_mrsh >> $chpcSMSInit
fi
if [[ ${enable_clustershell} -eq 1 ]];then
   # update clustershell for sms node
   cat $CHPC_SCRIPTDIR/sms/update_clustershell >> $chpcSMSInit
fi
\end{lstlisting}




%%% % begin_ohpc_run
%%% % ohpc_comment_header Add InfiniBand support services on master node \ref{sec:add_ofed}
%%% \begin{lstlisting}[language=bash,keywords={}]
%%% [sms](*\#*) (*\groupinstall*) "InfiniBand Support"
%%% [sms](*\#*) (*\install*) infinipath-psm
%%% 
%%% # Load IB drivers
%%% [sms](*\#*) systemctl start rdma
%%% \end{lstlisting}
%%% % end_ohpc_run
%%% 
%%% With the \InfiniBand{} drivers included, you can also enable (optional) IPoIB functionality
%%% which provides a mechanism to send IP packets over the IB network. If you plan
%%% to mount a \Lustre{} file system over \InfiniBand{} (see \S\ref{sec:lustre_client}
%%% for additional details), then having IPoIB enabled is a requirement for the
%%% \Lustre{} client. \OHPC{} provides a template configuration file to aid in setting up
%%% an {\em ib0} interface on the {\em master} host. To use, copy the template
%%% provided and update the \texttt{\$\{sms\_ipoib\}} and
%%% \texttt{\$\{ipoib\_netmask\}} entries to match local desired settings (alter ib0
%%% naming as appropriate if system contains dual-ported or multiple HCAs). 
%%% 
%%% % begin_ohpc_run
%%% % ohpc_validation_newline
%%% % ohpc_command if [[ ${enable_ipoib} -eq 1 ]];then
%%% % ohpc_indent 5
%%% % ohpc_validation_comment Enable ib0
%%% \begin{lstlisting}[language=bash,literate={-}{-}1,keywords={},upquote=true]
%%% [sms](*\#*) cp /opt/ohpc/pub/examples/network/centos/ifcfg-ib0 /etc/sysconfig/network-scripts
%%% 
%%% # Define local IPoIB address and netmask
%%% [sms](*\#*) perl -pi -e "s/master_ipoib/${sms_ipoib}/" /etc/sysconfig/network-scripts/ifcfg-ib0
%%% [sms](*\#*) perl -pi -e "s/ipoib_netmask/${ipoib_netmask}/" /etc/sysconfig/network-scripts/ifcfg-ib0
%%% 
%%% # Initiate ib0
%%% [sms](*\#*) ifup ib0
%%% \end{lstlisting}
%%% % ohpc_indent 0
%%% % ohpc_command fi
%%% % end_ohpc_run



\vspace*{-0.15cm}
\vspace*{-0.50cm}

\clearpage
\section{Instantiating OpenHPC System in OpenStack Cloud}
To instantiate OpenHPC system, we will first prepare openstack components with HPC images, networking and other relevant configurations. After the configuration we will instantiate HPC head node and HPC compute node using nova. 

It is assumed that system admin has installed OpenStack controller services and OpenStack network services (i.e. keystone, nova, ironic, glance, neutron, mongodb, rabbitmq server, heat etc). 
Controller node is configured with OpenVSwitch Bridge on internal network port. 
Two tenants, named \texttt{admin} and \texttt{services} are created in keystone to manage the services. All the services are created by system admin. 

Below is expected endpoint list.

% begin_ohpc_run
% ohpc_validation_comment #   XFILEX
% end_ohpc_run

% begin_ohpc_run
% ohpc_command #!/bin/bash
% ohpc_validation_comment #   FILE: deploy_chpc_openstack part 1

\begin{lstlisting}[language=bash,keywords={}]
[ctrlr](*\#*) openstack service list

#Expected Output#
+----------------------------------+-----------+---------------+---------------+
| ID                                         | Region    | Service Name  | Service Type  |
+----------------------------------+-----------+---------------+---------------+
| d5aeeb54713745c29ed3c2e4a97f59bd | RegionOne | ironic        | baremetal     |
| 86c71badbf8b4446a1b699eef05f3f41 | RegionOne | nova          | compute       |
| 70d138db26214d0bbc6b3ade8bf6f6f8 | RegionOne | gnocchi       | metric        |
| f34c3a58b9c648aaacabeeefd589a0d2 | RegionOne | neutron       | network       |
| 789c3fb6f9ae4e249ee4023484ccb5fc | RegionOne | aodh          | alarming      |
| 2531392e2d084b4582b364572e79a7b5 | RegionOne | heat          | orchestration |
| c183b73f654e454eaf5784c4b98149d8 | RegionOne | Image Service | image         |
| 850b3c2943df4dca99338ff2013f657b | RegionOne | cinder        | volume        |
| 81cefa79212a4780abe5a1da281a0172 | RegionOne | novav3        | computev3     |
| 36a6a7c7968a4a94bea07c8e30fa5c4b | RegionOne | keystone      | identity      |
| db70a8676dd44dd09b3ada7475e67383 | RegionOne | cinderv3      | volumev3      |
| c4161d4c9c6b4080b2cb66c2f580853d | RegionOne | ceilometer    | metering      |
| d5714e8adb094671ad0388d04214c44d | RegionOne | cinderv2      | volumev2      |
+----------------------------------+-----------+---------------+---------------+

[ctrlr](*\#*) openstack project list

#Expected Output#
+----------------------------------+----------+
| ID                                          | Name     |
+----------------------------------+----------+
| 7464fcc8f1b34048bd09fe165d18647b | admin    |
| b1ed7efb53cc44c8b06daaee15b6a296 | services |
+----------------------------------+----------+
\end{lstlisting}
% end_ohpc_run

Recipe below is tested with controller node installed and configured using packstack.
Reference section provide more detail on packstack installation of OpenStack.
Before starting with HPC instantiation, please export openstack credentials, we will be using them during openstack configuration. 

You can do this manually, as such:

\begin{lstlisting}[language=bash,keywords={}]
[ctrlr](*\#*) unset OS_SERVICE_TOKEN
[ctrlr](*\#*) export OS_USERNAME=admin
[ctrlr](*\#*) export OS_PASSWORD=<>
[ctrlr](*\#*) export OS_AUTH_URL=<>
[ctrlr](*\#*) export PS1='[\u@\h \W(keystone_admin)]\$ '
[ctrlr](*\#*) 
[ctrlr](*\#*) export OS_TENANT_NAME=admin
[ctrlr](*\#*) export OS_REGION_NAME=<>  
\end{lstlisting}

OR, if you've deployed via PackStack, you can just source the keystonerc_admin file. These credentials will be used throughout the rest of this document. If you have an existing Openstack installation with more complex credentials, you will need to set them per your configuration.

% begin_ohpc_run
\begin{lstlisting}[language=bash,keywords={}]
[ctrlr](*\#*) source ${HOME}/keystonerc_admin
\end{lstlisting}
% end_ohpc_run
	
\clearpage
\subsection{Prepare OpenStack for bare metal provisioning with ironic} \label{sec:o-s_prep-ironic}
This section we will create generic configuration, required for baremetal provisioning. We will use ironic as a provisioner and nova as a scheduler.
Have selinux in permissive mode
Setenforce 0

Create baremetal admin and baremetal observer role, and restart ironic API

\begin{lstlisting}[language=bash,keywords={}]
openstack role list | grep -i baremetal_admin
role_exists=$?
if [ "${role_exists}" -ne "0" ]; then 
    openstack role create baremetal_admin
fi

openstack role list | grep -i baremetal_observer 
role_exists=$?
if [ "${role_exists}" -ne "0" ]; then
    openstack role create baremetal_observer
fi
systemctl restart openstack-ironic-api
\end{lstlisting}

Install tftp and other packages required for pxe boot via ironinc
Ensure the utilities for baremetal are installed

\begin{lstlisting}[language=bash,keywords={}]
yum install -y tftp-server syslinux-tftpboot xinetd
\end{lstlisting}

Make the directory for tftp and give it the ironic owner

\begin{lstlisting}[language=bash,keywords={}]
mkdir -p /tftpboot
chown -R ironic /tftpboot
\end{lstlisting}

Configure tfpt server

\begin{lstlisting}[language=bash,keywords={}]
(*\#*)Configure tftp 
(*\#*)Configure /etc/xinet.d/tftp
echo "service tftp" > /etc/xinetd.d/tftp
echo "{" >> /etc/xinetd.d/tftp
echo "  protocol        = udp" >> /etc/xinetd.d/tftp
echo "  port            = 69" >> /etc/xinetd.d/tftp
echo "  socket_type     = dgram" >> /etc/xinetd.d/tftp
echo "  wait            = yes" >> /etc/xinetd.d/tftp
echo "  user            = root" >> /etc/xinetd.d/tftp
echo "  server          = /usr/sbin/in.tftpd" >> /etc/xinetd.d/tftp
echo "  server_args     = -v -v -v -v -v --map-file /tftpboot/map-file /tftpboot" >> /etc/xinetd.d/tftp
echo "  disable         = no" >> /etc/xinetd.d/tftp
echo "  (*\#*) This is a workaround for Fedora, where TFTP will listen only on" >> /etc/xinetd.d/tftp
echo "  (*\#*) IPv6 endpoint, if IPv4 flag is not used." >> /etc/xinetd.d/tftp
echo "  flags           = IPv4" >> /etc/xinetd.d/tftp
echo "}" >> /etc/xinetd.d/tftp

(*\#*)Restart the xinetd service
systemctl restart xinetd
    
(*\#*)Copy the PXE linux files to the tftpboot directory we created
cp /var/lib/tftpboot/pxelinux.0 /tftpboot
cp /var/lib/tftpboot/chain.c32 /tftpboot
    
(*\#*)Generate a map file for the PXE files
echo 're ^(/tftpboot/) /tftpboot/\2' > /tftpboot/map-file
echo 're ^/tftpboot/ /tftpboot/' >> /tftpboot/map-file
echo 're ^(^/) /tftpboot/\1' >> /tftpboot/map-file
echo 're ^([^/]) /tftpboot/\1' >> /tftpboot/map-file
\end{lstlisting}


Update Ironic configuration with tftp information. First update controller IP address for tftp server in ironic configuration.

\begin{lstlisting}[language=bash,keywords={}]
sed --in-place "s|(*\#*)tftp_server=\$my_ip|tftp_server=${controller_ip}|" /etc/ironic/ironic.conf
\end{lstlisting}

Update other additional tftp settings in ironic configuration file:


\begin{lstlisting}[language=bash,keywords={}]
sed --in-place "s|(*\#*)tftp_root=/tftpboot|tftp_root=/tftpboot|" /etc/ironic/ironic.conf
sed --in-place "s|(*\#*)ip_version=4|ip_version=4|" /etc/ironic/ironic.conf
sed --in-place "s|(*\#*)automated_clean=true|automated_clean=false|" /etc/ironic/ironic.conf
\end{lstlisting}

Now inform Nova to use ironic for bare metal provisioning, by configuring NOVA.conf

\begin{lstlisting}[language=bash,keywords={}]
sed --in-place "s|(*\#*)scheduler_use_baremetal_filters=false|scheduler_use_baremetal_filters=true|" \
 /etc/nova/nova.conf
\end{lstlisting}

In our sample, we will not use controller node for any compute resource so, lets mark reserved host memory as 0.

\begin{lstlisting}[language=bash,keywords={}]
sed --in-place "s|reserved_host_memory_mb=512|reserved_host_memory_mb=0|" /etc/nova/nova.conf
sed --in-place "s|(*\#*)scheduler_host_subset_size=1|scheduler_host_subset_size=9999999|" /etc/nova/nova.conf
\end{lstlisting}

For cloud-init we need to enable meta data server, which is done via neutron configuration.

\begin{lstlisting}[language=bash,keywords={}]
(*\#*) Enable meta data
(*\#*) Edit /etc/neutron/dhcp_agent.ini
sed --in-place "s|enable_isolated_metadata\ =\ False|enable_isolated_metadata\ =\ True|" \
 /etc/neutron/dhcp_agent.ini
sed --in-place "s|(*\#*)force_metadata\ =\ false|force_metadata\ =\ True|" \ /etc/neutron/dhcp_agent.ini
\end{lstlisting}

We will enable internal dns server to assign host name to the instances as requested by user. 

\begin{lstlisting}[language=bash,keywords={}]
(*\#*)(*\#*)(*\#*)(*\#*)(*\#*)
(*\#*) Enable internal dns for hostname resolution, if it already not set
(*\#*) manipulating configuration file via shell, alternate is to use openstack-config (TODO)
(*\#*)(*\#*)(*\#*)(*\#*)
(*\#*) setup dns domain first
if grep -q "^dns_domain.*openstacklocal$" /etc/neutron/neutron.conf; then
   sed -in-place  "s|^dns_domain.*|dns_domain = oslocal|" /etc/neutron/neutron.conf
else
   if ! grep -q "^dns_domain" neutron.conf; then
       sed -in-place  "s|^(*\#*)dns_domain = openstacklocal$|dns_domain = oslocal|" /etc/neutron/neutron.conf
   fi
fi
(*\#*) configure ml2 dns driver for neutron
ml2file=/etc/neutron/plugins/ml2/ml2_conf.ini
if ! grep -q "^extension_drivers" $ml2file; then
    (*\#*) Assuming there is a place holder in comments, replace that string
    sed -in-place  "s|^(*\#*)extension_drivers.*|extension_drivers = port_security,dns|" $ml2file
else
    (*\#*) Entry is present, check if dns is already present, if not then enable
    if ! grep "^extension_drivers" $ml2file|grep -q dns; then
        current_dns=`grep "^extension_drivers" $ml2file`
        new_dns="$current_dns,dns"
        sed -in-place  "s|^extension_drivers.*|$new_dns|" $ml2file
    fi
fi
\end{lstlisting}

We are pretty much done with initial configuration, so let’s restart all the services at controller node.

\begin{lstlisting}[language=bash,keywords={}]
systemctl restart neutron-dhcp-agent
systemctl restart neutron-openvswitch-agent
systemctl restart neutron-metadata-agent
systemctl restart neutron-server
systemctl restart openstack-nova-scheduler
systemctl restart openstack-nova-compute
systemctl restart openstack-ironic-conductor
\end{lstlisting}


\vspace*{-0.15cm}
\subsection{Instantiate bare metal nodes} \label{sec:instantiate-bare-metal}
\subsubsection{Setup generic bare metal instance}

This section configure open stack for bare metal instance according to HPC images and user inputs. Before starting this it is assumed that system administrator has installed openstack and its services and has done appropriate configuration for bare metal provisioning, which includes installing ironinc, keystone, nova, neutron, glance. It is assumed that keystone is configured with 

Before instantiating bare metal nodes with HPC, we need to do little bit more configuration. 
Setup generic bare metal instance

This section configures the network for "HPC as a services", upload compute OS images to glance, create a flavor for bare metal and upload public keys for ssh session.

First create a generic network for "HPC as a service" with a name "sharednet1"

\begin{lstlisting}[language=bash,keywords={}]
(*\#*)Get the tenant ID for the services tenant
    SERVICES_TENANT_ID=`keystone tenant-list | grep "|\s*services\s*|" | awk '{print $2}'`

    (*\#*)Create the flat network on which you are going to launch instances
    neutron net-list | grep "|\s*sharednet1\s*|"
    net_exists=$?
    if [ "${net_exists}" -ne "0" ]; then
        neutron net-create --tenant-id ${SERVICES_TENANT_ID} sharednet1 --shared \
         --provider:network_type flat --provider:physical_network physnet1
    fi
    NEUTRON_NETWORK_UUID=`neutron net-list | grep "|\s*sharednet1\s*|" | awk '{print $2}'`
\end{lstlisting}


Create a subnet for our cluster with user defined start and end IP addresses. Make the controller as a gateway for our instances.

\begin{lstlisting}[language=bash,keywords={}]
(*\#*)Create the subnet on the newly created network
    neutron subnet-list | grep "|\s*subnet01\s*|"
    subnet_exists=$?
    if [ "${subnet_exists}" -ne "0" ]; then
        neutron subnet-create sharednet1 --name subnet01 --ip-version=4 \
         --gateway=${controller_ip} --allocation-pool \
          start=${cc_subnet_dhcp_start},end=${cc_subnet_dhcp_end} --enable-dhcp \
           ${cc_subnet_cidr}
    fi
    NEUTRON_SUBNET_UUID=`neutron subnet-list | grep "|\s*subnet01\s*|" | awk '{print $2}'`
\end{lstlisting}

Upload kernel and initrd images to glance service so that they are available to ironic while deploying node.

\begin{lstlisting}[language=bash,keywords={}]
(*\#*)Create the deploy-kernel and deploy-initrd images
    glance image-list | grep "|\s*deploy-vmlinuz\s*|"
    img_exists=$?
    if [ "${img_exists}" -ne "0" ]; then
        glance image-create --name deploy-vmlinuz --visibility public --disk-format \
        aki --container-format aki < ${chpc_image_deploy_kernel}
    fi
    DEPLOY_VMLINUZ_UUID=`glance image-list | grep "|\s*deploy-vmlinuz\s*|" | awk '{print $2}'`

    glance image-list | grep "|\s*deploy-initrd\s*|"
    img_exists=$?
    if [ "${img_exists}" -ne "0" ]; then
        glance image-create --name deploy-initrd --visibility public --disk-format \
        ari --container-format ari < ${chpc_image_deploy_ramdisk}
    fi
    DEPLOY_INITRD_UUID=`glance image-list | grep "|\s*deploy-initrd\s*|" | awk '{print $2}
\end{lstlisting}

Create a bare metal flavor with nova.

\begin{lstlisting}[language=bash,keywords={}]
(*\#*)Create the baremetal flavor and set the architecture to x86_64
    (*\#*) This will create common baremetal flavor, if SMS node & compute has different
    (*\#*) characteristic than user shall create multiple flavor one each characterisitc
    nova flavor-list | grep "|\s*baremetal-flavor\s*|"
    flavor_exists=$?
    if [ "$flavor_exists" -ne "0" ]; then
        nova flavor-create baremetal-flavor baremetal-flavor ${RAM_MB} ${DISK_GB} ${CPU}
        nova flavor-key baremetal-flavor set cpu_arch=$ARCH
    fi
    FLAVOR_UUID=`nova flavor-list | grep "|\s*baremetal-flavor\s*|" | awk '{print $2}'`
(*\#*)Increase the Quota limit for admin to allow nova boot
    openstack quota set --ram 512000 --cores 1000 --instances 100 admin
\end{lstlisting}

Finally register public ssh keys with nova, so that admin can ssh to the node.

\begin{lstlisting}[language=bash,keywords={}]
(*\#*)Register SSH keys with Nova
 nova keypair-list | grep "|\s*ostack_key\s*|"
 keypair_exists=$?
 if [ "${keypair_exists}" -ne "0" ]; then
    nova keypair-add --pub-key ${HOME}/.ssh/id_rsa.pub ostack_key
 fi
\end{lstlisting}

Export keypay name for use it later in other sections

\begin{lstlisting}[language=bash,keywords={}]
    KEYPAIR_NAME=ostack_key
\end{lstlisting}

\subsubsection{Setup HPC head node}


Previous section we created generic bare metal setup. In this section we will create configuration for HPC head node in an OpenStack cloud.
We created HPC head node OS images in previous sections, let's upload this image to glance, and store IMAGE id in environment variable SMS\_DISK\_IMAGE\_UUID, to be used during boot. 

\begin{lstlisting}[language=bash,keywords={}]
(*\#*) Create sms node image
   glance image-list | grep "|\s*sms-image\s*|"
   img_exists=$?
   if [ "${img_exists}" -ne "0" ]; then
       glance image-create --name sms-image --visibility public --disk-format \
       qcow2 --container-format bare < ${chpc_image_sms}
   fi
   SMS_DISK_IMAGE_UUID=`glance image-list | grep "|\s*sms-image\s*|" | awk '{print $2}'`
\end{lstlisting}

For provisioning sms node with ironic, we need to register node with ironic. This is done by registering node's BMC, node characteristic (aka flavor) like memory, cpu, disk space and node architecture. And registering kernel boot images. We will use pxe\_ipmitool as a provisioning driver in ironic with a boot mode as bios.

\begin{lstlisting}[language=bash,keywords={}]
(*\#*)Create a sms node in the bare metal service ironic.
    ironic node-list | grep "|\s*${sms_name}$\s*|"
    node_exists=$?
    if [ "${node_exists}" -ne "0" ]; then 
        ironic node-create -d pxe_ipmitool -i deploy_kernel=${DEPLOY_VMLINUZ_UUID} -i \
         deploy_ramdisk=${DEPLOY_INITRD_UUID} -i ipmi_terminal_port=8023 -i \
          ipmi_address=${sms_bmc} -i ipmi_username=${sms_bmc_username} -i \
           ipmi_password=${sms_bmc_password} -p cpus=${CPU} -p memory_mb=${RAM_MB} -p \
         local_gb=${DISK_GB} -p cpu_arch=${ARCH} -p capabilities="boot_mode:bios" \
         -n ${sms_name}
    fi
    SMS_UUID=`ironic node-list | grep "|\s*${sms_name}\s*|" | awk '{print $2}'`
\end{lstlisting}

Now we need tell ironic about the network port on which node will perform pxe boot by configuring MAC Address. 

	\begin{lstlisting}[language=bash,keywords={}]
    (*\#*)Add the associated port(s) MAC address to the created node(s)
    ironic port-create -n ${SMS_UUID} -a ${sms_mac}
	\end{lstlisting}

Add the instance info and disk space for root 
Add the instance\_info/image\_source and instance\_info/root\_gb
    
    \begin{lstlisting}[language=bash,keywords={}]
    ironic node-update $SMS_UUID add instance_info/image_source=${SMS_DISK_IMAGE_UUID} \
     instance_info/root_gb=50
	\end{lstlisting}


We will assign a fixed IP address to sms node. This is done by associating sms node’s MAC address with neutron port. We will store this information in the neutron with sms\_name. we will also set environment SMS\_PORT\_ID variable with this port id, to be used during boot.
    
    \begin{lstlisting}[language=bash,keywords={}]
    (*\#*)Setup neutron port for static IP addressing of sms node, this is an optional part
    neutron port-create sharednet1 --dns_name $sms_name --fixed-ip ip_address=$sms_ip \
     --name $sms_name --mac-address $sms_mac
    SMS_PORT_ID=`neutron port-list | grep "|\s*$sms_name\s*|" | awk '{print $2}'`
	\end{lstlisting}


\subsubsection{Setup HPC compute nodes}

In previous section we configured Openstack to instantiate sms node. In this section we will be configuring openstack to instantiate HPC compute nodes.

For HPC compute nodes, we created compute node images, upload hpc compute node image to glance as a user image, and store IMAGE id in environment variable USER\_DISK\_IMAGE\_UUID, to be used during boot.

\begin{lstlisting}[language=bash,keywords={}]
(*\#*)Create the whole-disk-image from the user's qcow2 file
    glance image-list | grep "|\s*user-image\s*|"
    img_exists=$?
    if [ "${img_exists}" -ne "0" ]; then
        glance image-create --name user-image --visibility public --disk-format qcow2 \
         --container-format bare < ${chpc_image_user}
    fi
    USER_DISK_IMAGE_UUID=`glance image-list | grep "|\s*user-image\s*|" | awk '{print $2}'`
\end{lstlisting}


Similar to sms node, create setup for all compute nodes including creating ironic node, associating node MAC address, adding instance information and assigning fix IP address. In our example we used 4 hpc compute nodes. to store the information in each OpenStack component we will assign compute node host name as a name, which is host name prefix (as chosen by user in inputs), followed by a node counter. 

\begin{lstlisting}[language=bash,keywords={}]
(*\#*) Setup Compute nodes
(*\#*) Note: if installed from the rpm, the following script is installed as setup_compute_nodes.sh 

    for ((i=0; i < ${num_ccomputes}; i++)); do
        (*\#*)(*\#*)Create compute nodes in the bare metal service
        ironic node-list | grep "|\s*${cnodename_prefix}$((i+1))\s*|"
        node_exists=$?
        if [ "${node_exists}" -ne "0" ]; then
            ironic node-create -d pxe_ipmitool -i deploy_kernel=${DEPLOY_VMLINUZ_UUID} -i \
             deploy_ramdisk=${DEPLOY_INITRD_UUID} -i ipmi_terminal_port=8023 -i \
              ipmi_address=${cc_bmc[$i]} -i ipmi_username=${cc_bmc_username} -i \
               ipmi_password=${cc_bmc_password} -p cpus=${CPU} -p memory_mb=${RAM_MB} -p \
                local_gb=${DISK_GB} -p cpu_arch=${ARCH} -p capabilities="boot_mode:bios" -n \
                 ${cnodename_prefix}$((i+1))
        fi
        NODE_UUID_CC[$i]=`ironic node-list | grep "|\s*${cnodename_prefix}$((i+1))\s*|" | \
        awk '{print $2}'`

        (*\#*) update for compute nodes node MAC
        ironic port-create -n ${NODE_UUID_CC[$i]} -a ${cc_mac[$i]}

        (*\#*)Add the instance_info/image_source and instance_info/root_gb
        ironic node-update ${NODE_UUID_CC[$i]} add \
         instance_info/image_source=${USER_DISK_IMAGE_UUID} instance_info/root_gb=50

        (*\#*)Setup neutron port for static IP addressing of compute nodes
        cn_name=${cnodename_prefix}$((i+1))
        neutron port-create sharednet1 --dns_name $cn_name --fixed-ip ip_address=${cc_ip[$i]} \
         --name $cn_name --mac-address ${cc_mac[$i]}
        NEUTRON_PORT_ID_CC[$i]=`neutron port-list | grep "|\s*${cnodename_prefix}$((i+1))\s*|" \
         | awk '{print $2}'`
    Done
\end{lstlisting}


 Ironic periodically sync with Nova with available nodes. Nova then updates its record for all available hosts. So before booting the node with Nova allow some time to sync ironic with it. 


\begin{lstlisting}[language=bash,keywords={}]
(*\#*) Wait for the Nova hypervisor-stats to sync with available Ironic resources
sleep 121
\end{lstlisting}

\subsubsection{Boot SMS node}

In previous section we completed the bare metal configuration. User can request any available baremetal nodes by specifying the flavor they want and image they want to boot node with. For bare metal we created a flavor with name baremetal-flavor, we will provide this to nova with a CLI option --flavor. In our situation we will request 1 bare metal node with a baremetal flavor (--flavor) and SMS node image to boot (--image).  We also would like to reserve the IP address of this node. 

In previous section (setup sms) we associated one of the nodes MAC address with IP address, we will request this from nova by indicating port-id we created earlier (port-id=\${SMS\_PORT\_ID}). 

In a previous section we also created cloud-init script for sms nodes. We will provide cloud-init script (chpcSMSInit) to nova CLI option --user-data. For cloud init we will use metadata server, which will be provided by "--meta role= option". We will provide sms public key with "--key-name" option. At the end we will give our node a name. This name will be a host name of booted bare metal node.

Before booting, save boot command to a script, which will useful later on if user wants to re-instantiate same node.

\begin{lstlisting}[language=bash,keywords={}]
(*\#*)Boot the sms node with nova. chpcInit is set from prepare_cloudInit
echo "nova boot --config-drive true --flavor ${FLAVOR_UUID} --image ${SMS_DISK_IMAGE_UUID} \
 --key-name ${KEYPAIR_NAME} --meta role=webservers --user-data=$chpcSMSInit --nic \
 port-id=${SMS_PORT_ID} ${sms_name}" > boot_sms
\end{lstlisting}

Issue a boot command to nova to boot a SMS node:

\begin{lstlisting}[language=bash,keywords={}]
nova boot --config-drive true --flavor ${FLAVOR_UUID} --image ${SMS_DISK_IMAGE_UUID} --key-name
 ${KEYPAIR_NAME} --meta role=webservers --user-data=$chpcSMSInit --nic port-id=${SMS_PORT_ID} \
 ${sms_name}
\end{lstlisting}


Wait around 15 seconds before we boot compute nodes. This will allow enough time to boot SMS node before compute nodes starts. 

\begin{lstlisting}[language=bash,keywords={}]
sleep 15
\end{lstlisting}

\subsubsection{Boot compute nodes}

Booting compute nodes are very similar to SMS nodes. In our case we will boot 4 compute nodes (as specified in user inputs. Host name of compute node will use prefix defined by cnodename\_prefix variable, followed by node counter. For compute node we will use compute node image (USER\_DISK\_IMAGE\_UUID) and compute node cloud-init script (chpcInit). 


\begin{lstlisting}[language=bash,keywords={}]
for ((i=0; i < ${num_ccomputes}; i++)); do
filename="cn$((i+1))"
echo "nova boot --config-drive true --flavor ${FLAVOR_UUID} --image ${USER_DISK_IMAGE_UUID} \
 --key-name ${KEYPAIR_NAME} --meta role=webservers --user-data=$chpcInit --nic \
  port-id=${NEUTRON_PORT_ID_CC[$i]} ${cnodename_prefix}$((i+1))" > boot_$filename
nova boot --config-drive true --flavor ${FLAVOR_UUID} --image ${USER_DISK_IMAGE_UUID} \
 --key-name ${KEYPAIR_NAME} --meta role=webservers --user-data=$chpcInit --nic \
  port-id=${NEUTRON_PORT_ID_CC[$i]} ${cnodename_prefix}$((i+1))
(*\#*)wait for 5 sec before booting other compute node
sleep 5
done
\end{lstlisting}


% begin_ohpc_run

% #  XFILEX prepare_chpc_openstack
% #!/bin/bash
% #---------------------------------------------------------------------------------
% #This script installs and configures ironic for baremetal provisioning on CentOS 7
% #Using the OpenStack-Mitaka release.
% #This relies on the packstack installation to happen first and the keystonerc_admin
% #file being in the user's home directory. It is assumed this script is run with
% #sudo permissions.
% #---------------------------------------------------------------------------------
% 
% #Set SELinux to permissive
% setenforce 0
% 
% #Source the keystonerc_admin file
% source ${HOME}/keystonerc_admin
% 
% #Create roles for the baremetal service. These can be used later to give special permissions to baremetal. This script will be using the defaults.
% openstack role list | grep -i baremetal_admin
% role_exists=$?
% if [ "${role_exists}" -ne "0" ]; then 
%     openstack role create baremetal_admin
% fi
% 
% openstack role list | grep -i baremetal_observer
% role_exists=$?
% if [ "${role_exists}" -ne "0" ]; then 
%     openstack role create baremetal_observer
% fi
% systemctl restart openstack-ironic-api
% 
% #Ensure the utilities for baremetal are installed
% yum install -y tftp-server syslinux-tftpboot xinetd
% #Make the directory for tftp and give it the ironic owner
% mkdir -p /tftpboot
% chown -R ironic /tftpboot
% 
% #Configure /etc/xinet.d/tftp
% echo "service tftp" > /etc/xinetd.d/tftp
% echo "{" >> /etc/xinetd.d/tftp
% echo "  protocol        = udp" >> /etc/xinetd.d/tftp
% echo "  port            = 69" >> /etc/xinetd.d/tftp
% echo "  socket_type     = dgram" >> /etc/xinetd.d/tftp
% echo "  wait            = yes" >> /etc/xinetd.d/tftp
% echo "  user            = root" >> /etc/xinetd.d/tftp
% echo "  server          = /usr/sbin/in.tftpd" >> /etc/xinetd.d/tftp
% echo "  server_args     = -v -v -v -v -v --map-file /tftpboot/map-file /tftpboot" >> /etc/xinetd.d/tftp
% echo "  disable         = no" >> /etc/xinetd.d/tftp
% echo "  # This is a workaround for Fedora, where TFTP will listen only on" >> /etc/xinetd.d/tftp
% echo "  # IPv6 endpoint, if IPv4 flag is not used." >> /etc/xinetd.d/tftp
% echo "  flags           = IPv4" >> /etc/xinetd.d/tftp
% echo "}" >> /etc/xinetd.d/tftp
% 
% #Restart the xinetd service
% systemctl restart xinetd
% 
% #Copy the PXE linux files to the tftpboot directory we created
% cp /var/lib/tftpboot/pxelinux.0 /tftpboot
% cp /var/lib/tftpboot/chain.c32 /tftpboot
% 
% #Generate a map file for the PXE files
% echo 're ^(/tftpboot/) /tftpboot/\2' > /tftpboot/map-file
% echo 're ^/tftpboot/ /tftpboot/' >> /tftpboot/map-file
% echo 're ^(^/) /tftpboot/\1' >> /tftpboot/map-file
% echo 're ^([^/]) /tftpboot/\1' >> /tftpboot/map-file
% 
% #Edit /etc/ironic/ironic.conf file with the <controller_ip> variable's value
% sed --in-place "s|#tftp_server=\$my_ip|tftp_server=${controller_ip}|" /etc/ironic/ironic.conf
% sed --in-place "s|#tftp_root=/tftpboot|tftp_root=/tftpboot|" /etc/ironic/ironic.conf
% sed --in-place "s|#ip_version=4|ip_version=4|" /etc/ironic/ironic.conf
% sed --in-place "s|#automated_clean=true|automated_clean=false|" /etc/ironic/ironic.conf
% 
% #Edit /etc/nova/nova.conf file
% sed --in-place "s|reserved_host_memory_mb=512|reserved_host_memory_mb=0|" /etc/nova/nova.conf
% sed --in-place "s|#scheduler_host_subset_size=1|scheduler_host_subset_size=9999999|" /etc/nova/nova.conf
% sed --in-place "s|#scheduler_use_baremetal_filters=false|scheduler_use_baremetal_filters=true|" /etc/nova/nova.conf
% 
% # Enable meta data
% # Edit /etc/neutron/dhcp_agent.ini
% sed --in-place "s|enable_isolated_metadata\ =\ False|enable_isolated_metadata\ =\ True|" /etc/neutron/dhcp_agent.ini
% sed --in-place "s|#force_metadata\ =\ false|force_metadata\ =\ True|" /etc/neutron/dhcp_agent.ini
% 
% #####
% # Enable internal dns for hostname resolution, if it already not set
% # manipulating configuration file via shell, alternate is to use openstack-config (TODO)
% ####
% # setup dns domain first
% if grep -q "^dns_domain.*openstacklocal$" /etc/neutron/neutron.conf; then
%    sed -in-place  "s|^dns_domain.*|dns_domain = oslocal|" /etc/neutron/neutron.conf
% else
%    if ! grep -q "^dns_domain" neutron.conf; then
%        sed -in-place  "s|^#dns_domain = openstacklocal$|dns_domain = oslocal|" /etc/neutron/neutron.conf
%    fi
% fi
% # configure ml2 dns driver for neutron
% ml2file=/etc/neutron/plugins/ml2/ml2_conf.ini
% if ! grep -q "^extension_drivers" $ml2file; then
%     # Assuming there is a place holder in comments, replace that string
%     sed -in-place  "s|^#extension_drivers.*|extension_drivers = port_security,dns|" $ml2file
% else
%     # Entry is present, check if dns is already present, if not then enable
%     if ! grep "^extension_drivers" $ml2file|grep -q dns; then
%         current_dns=`grep "^extension_drivers" $ml2file`
%         new_dns="$current_dns,dns"
%         sed -in-place  "s|^extension_drivers.*|$new_dns|" $ml2file
%     fi 
% fi
% #------
% 
% # Now restart the services
% #Restart ironic, nova, neutron, and ovs to load in the new configuration
% systemctl restart neutron-dhcp-agent
% systemctl restart neutron-openvswitch-agent
% systemctl restart neutron-metadata-agent
% systemctl restart neutron-server
% systemctl restart openstack-nova-scheduler
% systemctl restart openstack-nova-compute
% systemctl restart openstack-ironic-conductor


% #  XFILEX prepare_chpc_image
% #!bin/bash
% set -x
% function setup_dib() {
%         #enable local disk-image-builder
%         # Check if disk-image-builder is installed
%         yum -y install diskimage-builder PyYAML
%         #install grub dependency
%         yum -y install parted
%         # copy patch to installed location
%         sudo cp -fr ../../dib/dib_patch/* /usr/share/diskimage-builder/
% }
% function setup_dib_hpc_base() {
%         # Install dib if it is not already installed
%         setup_dib
%         # For Debugging enable user
%         #export PATH=/home/ppk/PPK/dib/dev/diskimage-builder/bin:/home/ppk/PPK/dib/dev/dib-utils/bin:$PATH
%         # if cloudinit does not work then we will use this user for debugging
%         export DIB_DEV_USER_USERNAME=chpc
%         export DIB_DEV_USER_PASSWORD=intel8086
%         export DIB_DEV_USER_PWDLESS_SUDO=1
%         # For debugging enable DEBUG_TRACE
%         #export DIB_DEBUG_TRACE=1
%         # Add our custom element path
%         export ELEMENTS_PATH="$(realpath ../../dib/hpc/elements)"
%         # path to hpc configuration files i.e. cloud.cfg
%         export DIB_HPC_FILE_PATH="$(realpath ../../dib/hpc/hpc-files/)"
%         # define base for image as ohpc or orch.
%         export DIB_HPC_BASE=${chpc_base}
% 
%         #Path to HPC base yum repo file
%         # We support either Intel HPC Orchestrator and OpenHPC
%         if [[ "${DIB_HPC_BASE}" == "orch" ]]; then
%             export DIB_YUM_REPO_CONF=/etc/yum.repos.d/HPC_Orchestrator.repo
%             # for orch define Packge path
%             # This file is used to install orch component inside image
%             export DIB_HPC_ORCH_PKG=${orch_iso_path}
%         else
%             # Install the OpenHPC rpm
%             yum -y install ${ohpc_pkg}	
%             export DIB_HPC_OHPC_PKG=${ohpc_pkg}
%         fi
%         DIB_HPC_ELEMENTS="hpc-env-base"
% }
% 
% function prepare_sms_image() {
%     if [[ ${chpc_create_new_image} -ne 1 ]] && [[ -s $chpc_image_sms ]]; then
%         # No need to create an image, image is provided by user
%         echo -n "Skiping cloud sms-image build, Image provided:"
%         echo "$chpc_image_sms"
%         CHPC_IMAGE_DEST=$CHPC_CLOUD_IMAGE_PATH/$(basename $chpc_image_sms)
%         if [[ ! -e $CHPC_IMAGE_DEST ]]; then
%             sudo cp $chpc_image_sms $CHPC_CLOUD_IMAGE_PATH
%         fi
%         chpc_image_sms=$CHPC_IMAGE_DEST
%     else
%         echo "Building new User Image"
%         # First setup diskimage-builder
%         setup_dib_hpc_base
% 
%         # tell to build sms node image
%         export DIB_HPC_IMAGE_TYPE=sms
%  
%         # enable Resource Manager
%         DIB_HPC_ELEMENTS+=" hpc-slurm"
%         
%         #add mrsh if it is enabled
%         if [[ ${enable_mrsh} -eq 1 ]];then
%            DIB_HPC_ELEMENTS+=" hpc-mrsh"
%         fi
%         # for sms node setup dev environment
%         export DIB_HPC_COMPILER="gnu"
%         export DIB_HPC_MPI="openmpi mvapich2"
%         export DIB_HPC_PERF_TOOLS="perf-tools"
%         export DIB_HPC_3RD_LIBS="serial-libs parallel-libs io-libs python-libs runtimes"
%         DIB_HPC_ELEMENTS+=" hpc-dev-env"
%         # build an image
%         echo "====================================================================="
%         echo "=== Preparing cloud-hpc user image =================================="
%         echo "====================================================================="
%         disk-image-create centos7 vm local-config dhcp-all-interfaces devuser selinux-permissive $DIB_HPC_ELEMENTS -o icloud-hpc-cent7-sms 
%         echo "====================================================================="
%         echo "=== User Image Creation complete ===================================="
%         echo "====================================================================="
%         # User Image is reday
%         chpc_image_sms="$( realpath icloud-hpc-cent7.qcow2)"
%         mkdir -p $CHPC_CLOUD_IMAGE_PATH 
%         mv -f $chpc_image_sms $CHPC_CLOUD_IMAGE_PATH 
%         chpc_image_sms=$CHPC_CLOUD_IMAGE_PATH/$(basename $chpc_image_sms)
%     fi 
% }
% function prepare_user_image() {
%     if [[ ${chpc_create_new_image} -ne 1 ]] && [[ -s $chpc_image_user ]]; then
%         # No need to create an image, image is provided by user
%         echo -n "Skiping cloud user-image build, Image provided:"
%         echo "$chpc_image_user"
%         CHPC_IMAGE_DEST=$CHPC_CLOUD_IMAGE_PATH/$(basename $chpc_image_user)
%         if [[ ! -e $CHPC_IMAGE_DEST ]]; then
%             sudo cp $chpc_image_user $CHPC_CLOUD_IMAGE_PATH
%         fi
%         chpc_image_user=$CHPC_IMAGE_DEST
%     else
%         echo "Building new User Image"
%         # First setup diskimage-builder
%         setup_dib_hpc_base
% 
%         # tell to build sms node image
%         export DIB_HPC_IMAGE_TYPE=compute
% 
%         # enable Resource Manager
%         DIB_HPC_ELEMENTS+=" hpc-slurm"
%         
%         # add mrsh if it is enabled
%         if [[ ${enable_mrsh} -eq 1 ]];then
%            DIB_HPC_ELEMENTS+=" hpc-mrsh"
%         fi
%         # build an image
%         echo "====================================================================="
%         echo "=== Preparing cloud-hpc user image =================================="
%         echo "====================================================================="
%         disk-image-create centos7 vm local-config dhcp-all-interfaces devuser selinux-permissive $DIB_HPC_ELEMENTS -o icloud-hpc-cent7 
%         echo "====================================================================="
%         echo "=== User Image Creation complete ===================================="
%         echo "====================================================================="
%         # User Image is reday
%         chpc_image_user="$( realpath icloud-hpc-cent7.qcow2)"
%         mkdir -p $CHPC_CLOUD_IMAGE_PATH 
%         mv -f $chpc_image_user $CHPC_CLOUD_IMAGE_PATH 
%         chpc_image_user=$CHPC_CLOUD_IMAGE_PATH/$(basename $chpc_image_user)
%     fi 
% }
% 
% function prepare_deploy_image() {
%     if [[ ${chpc_create_new_image} -ne 1 ]] && [[ -s $chpc_image_deploy_kernel ]] && [[ -s $chpc_image_deploy_ramdisk ]]; then
%         # No need to create an image, image is provided by user
%         echo "Skiping cloud deploy-image build, Image provided:"
%         echo "Deploy kernel Image:$chpc_image_deploy_kernel"
%         echo "Deploy ramdisk Image:$chpc_image_deploy_ramdisk"
%         #Store Images file
%         CHPC_IMAGE_DEST=$CHPC_CLOUD_IMAGE_PATH/$(basename $chpc_image_deploy_kernel)
%         if [[ ! -e $CHPC_IMAGE_DEST ]]; then
%             sudo cp -f $chpc_image_deploy_kernel $CHPC_CLOUD_IMAGE_PATH/
%         fi
%         chpc_image_deploy_kernel=$CHPC_IMAGE_DEST
%         CHPC_IMAGE_DEST=$CHPC_CLOUD_IMAGE_PATH/$(basename $chpc_image_deploy_ramdisk)
%         if [[ ! -e $CHPC_IMAGE_DEST ]]; then
%             sudo cp -f $chpc_image_deploy_ramdisk $CHPC_CLOUD_IMAGE_PATH/
%         fi
%         chpc_image_deploy_ramdisk=$CHPC_IMAGE_DEST
%     else
%         echo "Building new Cloud Deploy Image"
%         echo "====================================================================="
%         echo "=== Preparing cloud-hpc deploy images for ironic====================="
%         echo "====================================================================="
%         #prepare deploy images
%         # Install dib if it is not already installed
%         setup_dib
%         # Unset any previos envirnment flag
%         unset DIB_YUM_REPO_CONF
%         #Install git if it is not already installed
%         yum -y install git
%         disk-image-create ironic-agent centos7 -o icloud-hpc-deploy-c7
%         echo "====================================================================="
%         echo "=== cloud-hpc deploy images Complete ================================"
%         echo "====================================================================="
%         chpc_image_deploy_kernel="$( realpath icloud-hpc-deploy-c7.kernel)"
%         chpc_image_deploy_ramdisk="$( realpath icloud-hpc-deploy-c7.initramfs)"
%         #Store Images file
%         mkdir -p $CHPC_CLOUD_IMAGE_PATH/
%         sudo mv -f $chpc_image_deploy_kernel $CHPC_CLOUD_IMAGE_PATH/
%         chpc_image_deploy_kernel=$CHPC_CLOUD_IMAGE_PATH/$(basename $chpc_image_deploy_kernel)
%         sudo mv -f $chpc_image_deploy_ramdisk $CHPC_CLOUD_IMAGE_PATH/
%         chpc_image_deploy_ramdisk=$CHPC_CLOUD_IMAGE_PATH/$(basename $chpc_image_deploy_ramdisk)
%     fi
% }
% 
% if [[ "${chpc_base}" == "orch" ]]; then 
%     CHPC_CLOUD_IMAGE_PATH=/opt/intel/hpc-orchestrator/admin/images/cloud/
% else
%     CHPC_CLOUD_IMAGE_PATH=/opt/ohpc/admin/images/cloud/
% fi
% 
% mkdir -p $CHPC_CLOUD_IMAGE_PATH
% ### Build HPC user image
% echo "########################################################################"
% echo "########################### Starting Image   ###########################"
% echo "########################################################################"
% prepare_sms_image
% echo $chpc_image_sms
% echo "########################################################################"
% echo "########################### sms image is done ##########################"
% echo "########################################################################"
% prepare_user_image
% echo $chpc_image_user
% echo "########################################################################"
% echo "########################### user image is done #########################"
% echo "########################################################################"
% #### Build hpc deploy image
% prepare_deploy_image
% echo $chpc_image_deploy_kernel
% echo $chpc_image_deploy_ramdisk
% echo "########################################################################"
% echo "########################### deploy image is done #######################"
% echo "########################################################################"

% #  XFILEX prepare_cloud_init
% #!bin/bash
% # 
% if [[ "${chpc_base}" == "orch" ]]; then
% 	echo "CloudInit: Intel Orchestrator"
%         chpcInitPath=/opt/intel/hpc-orchestrator/admin/cloud_hpc_init
% else
% 	echo "CloudInit: OpenHPC - ${chpc_base}"
%         chpcInitPath=/opt/ohpc/admin/cloud_hpc_init
% fi
% 
% # if directory exists then mv to Old directory. TBD
% mkdir -p $chpcInitPath
% #copy Cloud HPC files to temp working directory
% #copy cloud-init file for compute nodes
% sudo cp -fr -L ${SCRIPTDIR}/cloud_hpc_init/${chpc_base}/* $chpcInitPath/
% export chpcInit=$chpcInitPath/chpc_init
% export chpcSMSInit=$chpcInitPath/chpc_sms_init
% 
% #update sms_ip in cloudInit scripts for compute nodes
% sudo sed -i -e "s/<sms_ip>/${sms_ip}/g" $chpcInit
% 
% #Update variables to chpc_sms_init
% sudo sed -i -e "s/<update_cnodename_prefix>/${cnodename_prefix}/g" $chpcSMSInit
% sudo sed -i -e "s/<update_num_ccomputes>/${num_ccomputes}/g" $chpcSMSInit
% sudo sed -i -e "s/<update_ntp_server>/${controller_ip}/g" $chpcSMSInit
% sudo sed -i -e "s/<update_sms_name>/${sms_name}/g" $chpcSMSInit
% 
% if [[ ${enable_mrsh} -eq 1 ]];then
%    # update mrsh for sms node
%    cat $CHPC_SCRIPTDIR/sms/update_mrsh >> $chpcSMSInit
% fi
% if [[ ${enable_clustershell} -eq 1 ]];then
%    # update clustershell for sms node
%    cat $CHPC_SCRIPTDIR/sms/update_clustershell >> $chpcSMSInit
% fi
% # Internal dns is enabled, So no need to create /etc/hosts file
% # Prepare hosts file for sms & compute nodes
% #export etc_hosts=$chpcInitPath/hosts
% #sms_info="$sms_ip   $sms_name"
% #cat $etc_hosts|grep "$sms_info"
% #sexists=$?
% #if [ "${sexists}" -ne "0" ]; then 
% #   echo "$sms_info" >> $etc_hosts
% #fi
% ## Assuming no DNS, So we have to update hosts file so that sms node can communicate to compute node
% ## Update CN
% #for ((i=0; i < ${num_ccomputes}; i++)); do
% #    node_info="${cc_ip[$i]}  ${cnodename_prefix}$((i+1))"
% #    cat $etc_hosts|grep "$node_info"
% #    sexists=$?
% #    if [ "${sexists}" -ne "0" ]; then 
% #        echo "$node_info" >> $etc_hosts
% #    fi
% #done


% #  XFILEX deploy_chpc_openstack
% #!/bin/bash
% 
% 
% source ${HOME}/keystonerc_admin
% 
% # Function for common common configuration
% function setup_baremetal() {
%     #Get the tenant ID for the services tenant
%     SERVICES_TENANT_ID=`keystone tenant-list | grep "|\s*services\s*|" | awk '{print $2}'`
%     
%     #Create the flat network on which you are going to launch instances
%     neutron net-list | grep "|\s*sharednet1\s*|"
%     net_exists=$?
%     if [ "${net_exists}" -ne "0" ]; then
%         neutron net-create --tenant-id ${SERVICES_TENANT_ID} sharednet1 --shared --provider:network_type flat --provider:physical_network physnet1
%     fi
%     NEUTRON_NETWORK_UUID=`neutron net-list | grep "|\s*sharednet1\s*|" | awk '{print $2}'`
%     
%     #Create the subnet on the newly created network
%     neutron subnet-list | grep "|\s*subnet01\s*|"
%     subnet_exists=$?
%     if [ "${subnet_exists}" -ne "0" ]; then
%         neutron subnet-create sharednet1 --name subnet01 --ip-version=4 --gateway=${controller_ip} --allocation-pool start=${cc_subnet_dhcp_start},end=${cc_subnet_dhcp_end} --enable-dhcp ${cc_subnet_cidr}
%     fi
%     NEUTRON_SUBNET_UUID=`neutron subnet-list | grep "|\s*subnet01\s*|" | awk '{print $2}'`
%     #Create the deploy-kernel and deploy-initrd images
%     glance image-list | grep "|\s*deploy-vmlinuz\s*|"
%     img_exists=$?
%     if [ "${img_exists}" -ne "0" ]; then
%         glance image-create --name deploy-vmlinuz --visibility public --disk-format aki --container-format aki < ${chpc_image_deploy_kernel}
%     fi
%     DEPLOY_VMLINUZ_UUID=`glance image-list | grep "|\s*deploy-vmlinuz\s*|" | awk '{print $2}'`
%     
%     glance image-list | grep "|\s*deploy-initrd\s*|"
%     img_exists=$?
%     if [ "${img_exists}" -ne "0" ]; then
%         glance image-create --name deploy-initrd --visibility public --disk-format ari --container-format ari < ${chpc_image_deploy_ramdisk}
%     fi
%     DEPLOY_INITRD_UUID=`glance image-list | grep "|\s*deploy-initrd\s*|" | awk '{print $2}'`
%     
%     #Create the baremetal flavor and set the architecture to x86_64
%     # This will create common baremetal flavor, if SMS node & compute has different
%     # characteristic than user shall create multiple flavor one each characterisitc
%     nova flavor-list | grep "|\s*baremetal-flavor\s*|"
%     flavor_exists=$?
%     if [ "$flavor_exists" -ne "0" ]; then
%         nova flavor-create baremetal-flavor baremetal-flavor ${RAM_MB} ${DISK_GB} ${CPU}
%         nova flavor-key baremetal-flavor set cpu_arch=$ARCH
%     fi
%     FLAVOR_UUID=`nova flavor-list | grep "|\s*baremetal-flavor\s*|" | awk '{print $2}'`
%     #Increase the Quota limit for admin to allow nova boot
%     openstack quota set --ram 512000 --cores 1000 --instances 100 admin
%     
%     #Register SSH keys with Nova
%     nova keypair-list | grep "|\s*ostack_key\s*|"
%     keypair_exists=$?
%     if [ "${keypair_exists}" -ne "0" ]; then
%     nova keypair-add --pub-key ${HOME}/.ssh/id_rsa.pub ostack_key
%     fi
%     
%     KEYPAIR_NAME=ostack_key
% }
% 
% # Configure SMS Node
% function setup_sms() {
%    # Create sms node image
%    glance image-list | grep "|\s*sms-image\s*|"
%    img_exists=$?
%    if [ "${img_exists}" -ne "0" ]; then
%        glance image-create --name sms-image --visibility public --disk-format qcow2 --container-format bare < ${chpc_image_sms}
%    fi
%    SMS_DISK_IMAGE_UUID=`glance image-list | grep "|\s*sms-image\s*|" | awk '{print $2}'`
% 
%     #Create a sms node in the bare metal service ironic.
%     ironic node-list | grep "|\s*${sms_name}$\s*|"
%     node_exists=$?
%     if [ "${node_exists}" -ne "0" ]; then
%         ironic node-create -d pxe_ipmitool -i deploy_kernel=${DEPLOY_VMLINUZ_UUID} -i deploy_ramdisk=${DEPLOY_INITRD_UUID} -i ipmi_terminal_port=8023 -i ipmi_address=${sms_bmc} -i ipmi_username=${sms_bmc_username} -i ipmi_password=${sms_bmc_password} -p cpus=${CPU} -p memory_mb=${RAM_MB} -p local_gb=${DISK_GB} -p cpu_arch=${ARCH} -p capabilities="boot_mode:bios" -n ${sms_name}
%     fi
%     SMS_UUID=`ironic node-list | grep "|\s*${sms_name}\s*|" | awk '{print $2}'`
% 
%     #Add the associated port(s) MAC address to the created node(s)
%     ironic port-create -n ${SMS_UUID} -a ${sms_mac}
% 
%     #Add the instance_info/image_source and instance_info/root_gb
%     ironic node-update $SMS_UUID add instance_info/image_source=${SMS_DISK_IMAGE_UUID} instance_info/root_gb=50
% 
%     #Setup neutron port for static IP addressing of sms node, this is an optional part
%     neutron port-create sharednet1 --dns_name $sms_name --fixed-ip ip_address=$sms_ip --name $sms_name --mac-address $sms_mac
%     SMS_PORT_ID=`neutron port-list | grep "|\s*$sms_name\s*|" | awk '{print $2}'`
% }
% 
% #Configure Compute Nodes
% function setup_cn() {
%     #Create the whole-disk-image from the user's qcow2 file
%     glance image-list | grep "|\s*user-image\s*|"
%     img_exists=$?
%     if [ "${img_exists}" -ne "0" ]; then% 
%         glance image-create --name user-image --visibility public --disk-format qcow2 --container-format bare < ${chpc_image_user}
%     fi
%     USER_DISK_IMAGE_UUID=`glance image-list | grep "|\s*user-image\s*|" | awk '{print $2}'`
% 
%     # Setup Compute nodes
%     for ((i=0; i < ${num_ccomputes}; i++)); do
%         ##Create compute nodes in the bare metal service
%         ironic node-list | grep "|\s*${cnodename_prefix}$((i+1))\s*|"
%         node_exists=$?
%         if [ "${node_exists}" -ne "0" ]; then
%             ironic node-create -d pxe_ipmitool -i deploy_kernel=${DEPLOY_VMLINUZ_UUID} -i deploy_ramdisk=${DEPLOY_INITRD_UUID} -i ipmi_terminal_port=8023 -i ipmi_address=${cc_bmc[$i]} -i ipmi_username=${cc_bmc_username} -i ipmi_password=${cc_bmc_password} -p cpus=${CPU} -p memory_mb=${RAM_MB} -p local_gb=${DISK_GB} -p cpu_arch=${ARCH} -p capabilities="boot_mode:bios" -n ${cnodename_prefix}$((i+1))
%         fi
%         NODE_UUID_CC[$i]=`ironic node-list | grep "|\s*${cnodename_prefix}$((i+1))\s*|" | awk '{print $2}'`
% 
%         # update for compute nodes node MAC
%         ironic port-create -n ${NODE_UUID_CC[$i]} -a ${cc_mac[$i]}
% 
%         #Add the instance_info/image_source and instance_info/root_gb
%         ironic node-update ${NODE_UUID_CC[$i]} add instance_info/image_source=${USER_DISK_IMAGE_UUID} instance_info/root_gb=50
% 
%         #Setup neutron port for static IP addressing of compute nodes
%         cn_name=${cnodename_prefix}$((i+1))
%         neutron port-create sharednet1 --dns_name $cn_name --fixed-ip ip_address=${cc_ip[$i]} --name $cn_name --mac-address ${cc_mac[$i]}
%         NEUTRON_PORT_ID_CC[$i]=`neutron port-list | grep "|\s*${cnodename_prefix}$((i+1))\s*|" | awk '{print $2}'`
%     done
% }
% 
% 
% function boot_sms() {
%     #Boot the sms node with nova. chpcInit is set from prepare_cloudInit
%     echo "nova boot --config-drive true --flavor ${FLAVOR_UUID} --image ${SMS_DISK_IMAGE_UUID} --key-name ${KEYPAIR_NAME} --meta role=webservers --user-data=$chpcSMSInit --nic port-id=${SMS_PORT_ID} ${sms_name}" > boot_sms
%     nova boot --config-drive true --flavor ${FLAVOR_UUID} --image ${SMS_DISK_IMAGE_UUID} --key-name ${KEYPAIR_NAME} --meta role=webservers --user-data=$chpcSMSInit --nic port-id=${SMS_PORT_ID} ${sms_name}
% }
% 
% function boot_cn() {
%     for ((i=0; i < ${num_ccomputes}; i++)); do
%         filename="cn$((i+1))"
%         echo "nova boot --config-drive true --flavor ${FLAVOR_UUID} --image ${USER_DISK_IMAGE_UUID} --key-name ${KEYPAIR_NAME} --meta role=webservers --user-data=$chpcInit --nic port-id=${NEUTRON_PORT_ID_CC[$i]} ${cnodename_prefix}$((i+1))" > boot_$filename
%         nova boot --config-drive true --flavor ${FLAVOR_UUID} --image ${USER_DISK_IMAGE_UUID} --key-name ${KEYPAIR_NAME} --meta role=webservers --user-data=$chpcInit --nic port-id=${NEUTRON_PORT_ID_CC[$i]} ${cnodename_prefix}$((i+1))
%         #wait for 5 sec 
%         sleep 5
%     done
% }
% 
% 
% 
% ##Create a sms node in the bare metal service
% #ironic node-list | grep "|\s*${sms_name}$\s*|"
% #node_exists=$?
% #if [ "${node_exists}" -ne "0" ]; then
% #    ironic node-create -d pxe_ipmitool -i deploy_kernel=${DEPLOY_VMLINUZ_UUID} -i deploy_ramdisk=${DEPLOY_INITRD_UUID} -i ipmi_terminal_port=8023 -i ipmi_address=${sms_bmc} -i ipmi_username=${sms_bmc_username} -i ipmi_password=${sms_bmc_password} -p cpus=${CPU} -p memory_mb=${RAM_MB} -p local_gb=${DISK_GB} -p cpu_arch=${ARCH} -p capabilities="boot_mode:bios" -n ${sms_name}
% #fi
% #SMS_UUID=`ironic node-list | grep "|\s*${sms_name}\s*|" | awk '{print $2}'`
% 
% 
% #### main
% # First setup baremetnal environment
% setup_baremetal
% # Setup sms node first
% setup_sms
% # Setup Compute nodes
% setup_cn
% # Wait for the Nova hypervisor-stats to sync with available Ironic resources
% sleep 121
% # Now start booting the nodes
% # Boot sms node first
% boot_sms
% # wait for 15 sec before starting to boot compute nodes. TBD need to tune this time
% # SMS node should be booted before compute nodes starts booting. At minimum
% # sms node shall have cloud init executed before CN's cloud init
% sleep 15
% # Now boot compute nodes
% boot_cn
% 
% #nova boot --config-drive true --flavor ${FLAVOR_UUID} --image ${USER_DISK_IMAGE_UUID} --key-name ${KEYPAIR_NAME} --meta role=webservers --user-data=$chpcSMSInit --nic port-id=${SMS_PORT_ID} ${sms_name}
% sleep 5
% # Boot CNs









% 
% end_ohpc_run

\end{document}
